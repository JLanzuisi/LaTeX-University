\input{../Plantillas-Fomato/Tareas/tarea.tex}
\tcabe{introducción a las probabilidades}{Jhonny Lanzuisi, 1510759}
\cabe{introducción a las probabilidades: tercer parcial}
\begin{document}
	\thispagestyle{plain}
\chapter*{TERCER PARCIAL}
\noindent Las preguntas siguen la numeración del parcial original.
\subsection{Respuesta a la primera pregunta}
Para obtener la distribución acumulada, queremos calcular la integral:
\[ \iint\limits_{-\infty}^{X} e^{-x}(1+e^{-x})^{-2} \mathrm{d}x. \]
Hagamos $u=1+e^{-x}$, entonces $\mathrm{d}u=-e^{-x}$ y nuestra integral se convierte en
\begin{align*}
	\int_{-\infty}^{X} e^{-x}(1+e^{-x})^{-2} \mathrm{d}x &= -\int_{-\infty}^{X} \frac{1}{u^2} du \\
														&= \left.\frac{1}{u}\right|_{-\infty}^X \\
														&= \left.\frac{1}{1+e^{-x}}\right|_{-\infty}^X \\
														&= \frac{1}{1+e^{-X}}.
\end{align*}
\subsection{Respuesta a la segunda pregunta}
Calculamos primero la distribución acumulada de $Y$. Notemos que 
\[ F_Y(x) = P\{ Y<y \} = P\{ e^X<y \} = P\{X<\log(y) \}. \]
Entonces
\[ F_Y(y) = \int_{-\infty}^{\log(y)} f(x) \mathrm{d}x \]
y derivando obtenemos que
\[ f_Y(y) = \frac{1}{y}\, f(\log(y)). \]
Y esto último es lo mismo que
\[ f_Y(y) = \frac{1}{\sqrt{2\pi}\tau} \exp\left(-\frac{\log(y)^2}{2\tau^2}\right) \frac{1}{y}. \]
\subsection{Respuesta a la cuarta pregunta}

 $F$ es una función de densidad si
	\[ \iint\limits_{-\infty}^{A} f(x,y) \mathrm{d}x\mathrm{d}y = \iint\limits_{-\infty}^{\infty} cye^{-xy} \mathrm{d}x\mathrm{d}y = 1. \]
	Veamos que
	\[ \int_{-\infty}^{\infty} cye^{-xy} \mathrm{d}x = \left. cy \frac{-e^{-yx}}{y}\right|_{-\infty}^{\infty} \]
\subsection{Respuesta a la quinta pregunta}
Como la esperanza es una función lineal, tenemos que
\[ E(8X-Y+12) = 8E(X) -E(Y) + 12 = \frac{8}{\theta} - \lambda + 12.\]
\subsection{Repuesta a la sexta pregunta}
La esperanza viene dada por
\[ E[X] = \int_0^\infty xf_X(x) \mathrm{d}x \]
donde $f_X(x)$ es la función de densidad de la distribución gamma. Tenemos entonces que
\begin{align*}
	E[X] &= \frac{\lambda^\alpha}{\Gamma(\alpha)} \int_0^\infty	x^\alpha e^{-\lambda x} \mathrm{d}x \\
		 &= \frac{\lambda^\alpha}{\Gamma(\alpha)} \int_0^\infty \left( \frac{t}{\lambda} \right)^\alpha e^t \frac{\mathrm{d}t}{\lambda} \quad (t=\lambda x)\\
		 &= \frac{\lambda^\alpha}{\lambda^{\alpha+1}\Gamma(\alpha)} \int_0^\infty t^\alpha e^{-t} \mathrm{d}t \\
		 &= \frac{\Gamma(\alpha+1)}{\lambda\Gamma(\alpha)} \\
		 &= \frac{\alpha\Gamma(\alpha)}{\lambda\Gamma(\alpha)} \\
		 &= \frac{\alpha}{\lambda}
\end{align*}
\subsection{Respuesta a la séptima pregunta}
La esperanza viene dada por
\[ E[X] = \int_0^1 xf_X(x) \mathrm{d}x \]
donde $f_X(x)$ es la función de densidad de la distribución beta. Tenemos entonces que
\begin{align*}
	E[X] &= \frac{1}{B(a,b)} \int_0^1 x^a (1-x)^{b-1} \\
		 &= \frac{B(a+1,b)}{B(a,b)} \\
		 &= \frac{\Gamma(a+1)\Gamma(b)}{\Gamma(a+b+1)} \frac{\Gamma(a+b)}{\Gamma(a)\Gamma(b)} \\
		 &= \frac{a}{a+b} \frac{\Gamma(a)\Gamma(b)\Gamma(a+b)}{\Gamma(a)\Gamma(b)\Gamma(a+b)} \\
		 &= \frac{a}{a+b}.
\end{align*}
\end{document}