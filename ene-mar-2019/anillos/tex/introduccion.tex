\thispagestyle{plain}
\chapter{Introducción}%
\label{cha:Introducción}

{\noindent Aunque no es estrictamente necesario, sería bueno tener algo de conocimiento sobre la teoría de grupos, en la cual se estudian las propiedades generales de un conjunto con una sola operación binaria.}

Nos interesa ahora pensar el caso de las estructuras algebraicas que poseen dos operaciones binarias, sus propiedades y sus aplicaciones. En este sentido, el lector no deberá sorprenderse al ver que comenzamos nuestro estudio de forma análoga al estudio de la teoría de grupos, hablando de subanillos, anillos cocientes, ideales (que son el equivalente de los subgrupos normales) y homomorfismos de anillos. En lo que concierne a la nueva operación binaria introducida, el \textit{producto}, su estudio nos llevará al concepto de Cuerpo y eventualmente a construcciones mas específicas como los cuerpos finitos.
