%\thispagestyle{plain} 
%\setcounter{page}{1} \pagenumbering{roman}
\chapter{Prefacio}%
\label{cha:Prefacio}


{\noindent Estas notas cumplen un doble propósito: ser de ayuda a cualquier infortunado que se tope con ellas y servir como material de estudio a su autor. }

Comenzando como una introducción a la teoría de Anillos, basado en el curso que se dicta en la USB\footnote{Universidad Simón Bolívar.}, estas notas han ido creciendo hasta convertirse en lo que son hoy: una introducción al álgebra abstracta, desde los conceptos más básicos de la teoría de conjuntos y relaciones hasta la teoría de anillos. 

Los prerrequisitos para leer el texto, siendo este una introducción, son mínimos. Dicho esto, no vendría mal haber visto alguna vez un poco de matemáticas `abstractas' o `rigurosas': como los argumentos de `epsilon-delta' que se dan en un curso usual de cálculo, o las demostraciones axiomáticas típicas de un curso sobre geometría euclídea. En cualquier caso, lo anterior no es en absoluto un requisito por lo que el lector que no esté acostumbrado a estos temas no debe temer: este libro también es para el.

\marginnote{Los comentarios en el texto estarán en este formato. El lector es libre de ignorarlos, a veces serán referencias históricas, otras veces sugerencias.}
%Con respecto al primero de estos propósitos, los contenidos serán mas o menos los del curso `Introducción a la teoría de anillos' que se da en la USB\footnote{Universidad Simón Bolívar.}. Si el lector no es de los que tienen la fortuna (o quizás la desgracia) de estar viendo este curso, su propósito educativo es muy claro: servir de introducción a la teoría de anillos. En este sentido, se presupone cierta comodidad con los conceptos de la teoría de grupos\footnote{Que se pueden consultar en el herstein\cite{topicsinalgebra}}.

Se podrán encontrar una variedad de ejercicios de distintas fuentes, los de ellos que estén resueltos, lo están con la intención de que el lector los intente por su cuenta antes de ver la resolución propuesta. Evidentemente las respuestas dadas no serán las mas elegantes, en cuyo caso se insta al lector a enviar aquellas resoluciones que el o ella\footnote{O como el lector prefiera identificarse, no quiero ser acusado de lógica binaria.} considere mejores que las aquí presentes, al correo \url{jalb97@gmail.com}. 
%
Por último, las referencias podrán encontrarse al final del texto. 
