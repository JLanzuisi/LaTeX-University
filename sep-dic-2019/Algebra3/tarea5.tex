\input{../../Plantillas-Fomato/Tareas/tarea.tex}
\cabe{Álgebra 3}{Jhonny Lanzuisi, 1510759}
\renewcommand{\labelenumi}{\upshape\scshape\roman{enumi}.}
\newcommand{\bigzero}{\mbox{\normalfont\LARGE $0$}} 
\begin{document}
	\tituloD{álgebra iii}{Tarea: Matrices Diagonales de Bloques}
	\begin{teo}
		Si $A=A_{11}\oplus A_{22}\oplus\cdots\oplus A_{mm}$ es una matriz bloque diagonal, entonces
		\begin{enumerate}
			\item El polinomio característico de $A$, $p_A$, esta dado por
			\[ p_A(t) = \prod_{i=1}^{m} p_{A_{ii}}(t). \]
			\item El polinomio minimal de $A$, $m_A$, esta dado por
			\[ m_A = \mcm(A_{11},\dots,A_{mm}).  \]
			\item Para cualquier polinomio $f$ se tiene que
			\[ f(A) = f(A_{11})\oplus f(A_{22})\oplus\cdots\oplus f(A_{mm}). \]
		\end{enumerate}
	\end{teo}
	\begin{proof}
		Para ver \textsc{(i)}, hagamos $p_A(t) = \det(A-tI_m)$ y supongamos que las matrices $A_{ii}$ son de tamaño $n_i\t n_i$ donde los $n_i$ son números positivos ($0\leq i\leq m$), luego
%		\begingroup
%		\setlength\arraycolsep{-3pt}
		\begin{align*}
			p_A(t) &=\det\big((A_{11} - tI_{n_1})\oplus(A_{22} - tI_{n_2})\oplus\cdots\oplus(A_{mm} - tI_{n_m})\big)  \\
			&=\det\big(A_{11} - tI_{n_1}\big)\cdots\det\big(A_{mm} - tI_{n_m}\big) \\
			&=p_{A_{11}}(t)p_{A_{22}}(t)\cdots p_{A_{mm}}(t).
		\end{align*}
	Consideremos (\textsc{iii}). Sea $f(t) = t^n+a_{n-1}t^{n-1}+\cdots+a_1t+a_0$ un polinomio en $t$; entonces, tomando en cuenta que las potencias de una matriz diagonal se obtienen elevando los elementos de la diagonal, tenemos
%	\begin{small}
		\begin{align*}
		f(A)  &= A^n+a_{n-1}A^{n-1}+\cdots+a_1A+a_0I \\
		&= (A_{11}^n\oplus\cdots\oplus A_{mm}^n) \\
		&\phantom{=}\hspace{6em} + a_{n-1}(A_{11}^{n-1}\oplus\cdots\oplus A_{mm}^{n-1})+\cdots \\
		&\phantom{=}\hspace{9em} +a_1(A_{11}\oplus\cdots\oplus A_{mm}) + a_0 I \\
		&=(A_{11}^n+a_{n-1}A_{11}^{n-1}+\cdots+a_1A_{11}+a_0I)\oplus\cdots\\
		&\phantom{=}\hspace{5em} \oplus(A_{mm}^n+a_{n-1}A_{mm}^{n-1}+\cdots+a_1A_{mm}+a_0I) \\
		&= f(A_{11})\oplus f(A_{22})\oplus\cdots\oplus f(A_{mm}).
		\end{align*}
%	\end{small}
%	\endgroup
	Consideremos ahora (\textsc{ii}). Llamemos $m_i$ al polinomio minimal asociado a $A_{ii}$ ($0\leq i\leq m$), entonces tenemos que $m_A(A_{ii}) = 0$ y $m_i$ divide a $m_a$ de donde se sigue $\mcm(A_{11},\dots,A_{mm})$ divide a $m_A$. Por otro lado, tenemos que $\mcm(A_{11},\dots,A_{mm})(A) = 0$ por lo que $m_A$ divide a $\mcm(A_{11},\dots,A_{mm})$ y se obtiene la igualdad $m_A = \mcm(A_{11},\dots,A_{mm})$.
	\end{proof}
\end{document}