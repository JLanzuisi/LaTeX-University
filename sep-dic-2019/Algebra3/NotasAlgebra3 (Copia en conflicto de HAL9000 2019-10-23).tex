\input{../../Plantillas-Fomato/Libros/NotasDeClases}

\begin{document}
\frontmatter

\begin{titlepage}
	\centering
	{\huge\addfontfeatures{LetterSpace=5} CLASES DE ÁLGEBRA \liningnums 3}\\[1em]
	{\Large{\scshape sep-dic} 2019}\\
	\vfill
	{\large Jhonny Lanzuisi, 15\,10759}\\[.7em]
	{\large\ttfamily jalb97@gmail.com}
	\vfill
	Universidad Simón Bolívar \hfill Caracas, Venezuela
\end{titlepage}

%\begingroup
%% a \clearpage will close the group and restore the meaning
%\let\clearpage\endgroup
%\tableofcontents
\begin{flushleft}
	\begin{tabular}[c]{rlr} 
		\addlinespace[.5em]
		\multicolumn{3}{l}{\LARGE Evaluaciones} \\[1em]
		{ 1}er Parcial: & Semana 6 &({40}\%) \\
		{ 2}do Parcial: & Semana 11 &({50}\%) \\
		 Problemarios: & Antes de parcial &({10}\%) \\
	\end{tabular}
\end{flushleft}
%\twocolumn
\mainmatter

\chapter[normas y productos internos]{Normas y Productos Internos}

\section{Normas}
\clase{1}{Martes}{17/12}
\noindent
\begin{defi}[norma]
	Sea $\ve V$ un espacio vectorial real o complejo. Una norma sobre $\ve V$ es una función $\norm\colon\ve V\to\R$ tal que, para todo $\Ve{x, y}\in\ve V$ y $\lambda$ un escalar,
	\begin{enumerate}
		\item $\Norm{\Ve x}>0$ y $\Norm{\Ve x} = 0$ si, y solo si, $\Ve x = 0$.
		\item $\Norm{\lambda\Ve x} = |\lambda|\Norm{\Ve x}$
		\item $\Norm{\Ve x+\Ve y} \leq \Norm{\Ve{x}} + \Norm{\Ve y}$.
	\end{enumerate}
\end{defi}

\noindent
En un espacio vectorial $\ve V$ que posee una norma se puede definir la noción de distancia entre dos puntos $\Ve{x,y}\in\ve V$ como $\Norm{\Ve x-\Ve y}$.

\begin{ejem}
	La \emph{norma canónica} de $\Rn$ esta dada, para todo $\Ve x=(\dt{x}{,})$, por
	\[ \Norm{\dt{x}{,}} = \sqrt{\dt{x^2}{+}}. \]
	El hecho de que esta norma cumple con las propiedades (1) y (2) de la definición anterior es inmediato debido a las propiedades de las raíces cuadradas reales. La propiedad (3) es menos evidente, la demostraremos mas adelante con ayuda del producto interno.
\end{ejem}

\begin{ejem}
	En $\Cn$ definimos:
	\[ \Norm{\dt{x}{,}} = \sqrt{|x_1|^2+\dots+|x_n|^2}. \]
	Al igual que antes, las propiedades (1) y (2) son consecuencia de las propeidades de las raices reales.
\end{ejem}

\begin{ejem}
	En $\Rn$ definimos:
	\[ \Norm{\dt{x}{,}} = \max(|x_1|,\dots,|x_n|). \]
	Veamos que esta norma cumple con las tres propiedades:
	\begin{enumerate}
		\item Dado que $|x_1|,\dots,|x_n|$ son todos positivos, el mayor de ellos también lo es. Si $\max(|x_1|,\dots,|x_n|) = 0$ es evidente que $x=0$, el recíproco es igual de fácil.
		\item Sea $|x_k|\,(1\leq k\leq n)$ el mayor de los $|x_1|,\dots,|x_n|$. Entonces
		\[ |cx_k| = |c||x_k| \geq |c||x_i| \quad\text{para todo $1\leq k\leq n$}. \]
		\item Consideremos el vector suma:
		\[ \Ve x+\Ve y = (x_1+y_1,\dots,x_n+y_n). \]
		Supongamos que $\max(\Ve x+\Ve y) = |x_k+y_j|$ con $1\leq k,j\leq n$. Entonces
		\[ |x_k+y_j|\leq|x_k|+|y_j|\leq\max(\Ve x) + \max(\Ve y). \]
	\end{enumerate}
\end{ejem}
\begin{ejem}
	En $\Rn$ definimos
	\[ \Norm{\dt{x}{,}} = \sum_{k=1}^{n} |x_k|. \]
\end{ejem}
\section{Producto interno}
\begin{defi}[producto interno]
	Sea $\ve V$ un espacio vectorial real o complejo. Un producto interno real sobre $\ve V$ es una función $\iprod\colon\ve V\t\ve V\to\R$ tal que, para todo $\Ve{x,y,z}\in\ve V$ y $\lambda$ un escalar,
	\begin{enumerate}
		\item $\Iprod{\Ve x}{\Ve x}>0$ y $\Iprod{\Ve x}{\Ve x} = 0$ si, y solo si, $\Ve x=0$.
		\item $\Iprod{\Ve x}{\Ve y} = \overline{\Iprod{\Ve y}{\Ve x}}$.
		\item $\Iprod{\lambda\Ve x}{\Ve y} = \lambda\Iprod{\Ve x}{\Ve y}$.
		\item $\Iprod{\Ve x+\Ve z}{\Ve y} = \Iprod{\Ve x}{\Ve y} + \Iprod{\Ve z}{\Ve y}$.
	\end{enumerate}
Si $\ve V$ es de dimensión finita lo llamaremos \emph{espacio euclídeo}.
\end{defi}

Nótese que las propiedades (2) y (3) implican
\begin{align*}
	\Iprod{\Ve x}{\lambda \Ve y} &= \overline{\Iprod{\lambda\Ve y}{\Ve x}} \\
						 		 &= \overline{\lambda}\,\overline{\Iprod{\Ve y}{\Ve x}} \\
						 		 &= \overline{\lambda}\Iprod{\Ve x}{\Ve y}.
\end{align*}
De forma similar las propiedades (2) y (4) implican
\begin{align*}
	\Iprod{\Ve x}{\Ve y+\Ve z} &= \overline{\Iprod{\Ve y+\Ve z}{\Ve x}} \\
							&= \overline{\Iprod{\Ve y}{\Ve x}} + \overline{\Iprod{\Ve z}{\Ve x}} \\
							&= \Iprod{\Ve x}{\Ve y} + \Iprod{\Ve x}{\Ve z}.
\end{align*}

Por todo lo anterior los productos internos son transformaciones lineales.

\begin{ejem}
	En $\Rn$ el \emph{producto interno canónico} esta dado por
	\[ \Iprod{\Ve x}{\Ve y} = \Iprod{(\dt{x}{,})}{(\dt{y}{,})} = \sum_{k=1}^{n} x_ky_k.\]
	Para este producto interno las propiedades (1)-(4) no son difíciles de verificar.
\end{ejem}

\clase{2}{Lunes}{23}
({\scshape nada nuevo})
\clase{3}{Martes}{24}
\begin{ejem}
	En $\Cn$ el \emph{producto interno canónico} esta dado por
	\[ \Iprod{\Ve x}{\Ve y} = \Iprod{(\dt{x}{,})}{(\dt{y}{,})} = \sum_{k=1}^{n} x_k\overline{y_k}.\]
	al igual que el ejemplo anterior las propiedades (1)-(4) no son difíciles de verificar.
\end{ejem}
\begin{ejem}
	En $\ve V= M_{mn}(\Co)$ definimos un producto interno como
	\[ \Iprod{A}{B} = \traz(AB^\ast) \]
	donde $A,B\in M_{mn}(\Co)$ y $B^\ast$ es la \emph{adjunta} (transpuesta de la conjugada) de $B$. Veamos que esta es en efecto un producto interno.
	Para la propiedad (1), notemos que
	\begin{align*}
		\Iprod{A}{A}  &= \traz(AA^\ast) \\
					  &= \sum_{i=1}^{m} AA^\ast_{(ii)} \\
					  &= \sum_{i=1}^{m}\sum_{k=1}^{n} A_{(ik)}A^\ast_{(ki)} \\
					  &= \sum_{i=1}^{m}\sum_{k=1}^{n} A_{(ik)}\overline{A}_{(ik)} \\
					  &= \sum_{i=1}^{m}\sum_{k=1}^{n} A^2_{(ik)}
	\end{align*}
	de donde es claro que $\Iprod{A}{A}>0$ y $\Iprod{A}{A} = 0$ si, y solo si, la suma de los $A_{(ik)}$ es cero, es decir, si $A=0$.
	
	Para (2), veamos que
	\begin{align*}
		\overline{\Iprod{A}{B}} &= \overline{\traz(AB^\ast)} \\
					 			&= \sum_{i=1}^{m}\sum_{k=1}^{n} \overline{A}_{(ik)}\overline{B^\ast}_{(ki)} \\
					 			&= \sum_{i=1}^{m}\sum_{k=1}^{n} \overline{A}_{(ik)}^\mathrm t(B^\mathrm t)^\mathrm t_{(ki)} \\
					 			&= \sum_{i=1}^{m}\sum_{k=1}^{n} B_{(ik)}A_{(ki)}^\ast
	\end{align*}
\end{ejem}
\chapter[formas bilineales y cuadráticas]{Formas Bilineales y Cuadráticas}

\chapter[teorema de cauchy-hamilton]{Teorema de Cauchy-Hamilton}

\chapter[forma canónica de jordan]{Forma Canónica de Jordan}

\chapter[teorema de descomposición cíclica]{Teorema de Descomposición Cíclica}

\backmatter
\end{document}