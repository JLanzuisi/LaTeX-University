\input{../../Plantillas/Tareas/tarea.tex}
\cabe{Álgebra 3: Tarea 3}{Jhonny Lanzuisi, 1510759}
\begin{document}
\chapter{Formas Cuadráticas}%
	\section*{Ejercicio 5.3.1}
	\marginnote{\sffamily Jhonny Lanzuisi,\\ 1510759}%
	Encuentre un polinomio cuadrático diagonal \% isométrico a:
	\begin{enumerate}
		\item $x^2-xy+y^2+xz-z^2$.
		\item $xy-xz-yz$.
		\item $xz+yw$.
	\end{enumerate}
\paragraph{solucion}
	\textsc{consideremos primero} el polinomio $x^2-xy+y^2+xz-z^2$. La matriz simétrica $\mathsf S$ asociada a el ha de cumplir que $s_{ij}=1/2(a_{ij}+a_{ji})$ donde los $a_{ij}$ son los coeficientes de nuestro polinimio cuadrático. Se sigue entonces que
	\[ S = \begin{pmatrix}
	\phantom{-}1 & -1/2 & \phantom{-}1/2 \\
	-1/2 & \phantom{-}1 & \phantom{-}0 \\
	\phantom{-}1/2 & \phantom{-}0 & -1
	\end{pmatrix}. \]
	Por lo que la forma bilineal asociada $\Bform_S$ viene dada, para $v=(a_1,a_2,a_3)$ y $w=(b_1,b_2,b_3)$, por
	\begin{align*}
	v^tAw &= (a_1,a_2,a_3) \begin{pmatrix}
	\phantom{-}1 & -1/2 & \phantom{-}1/2 \\
	-1/2 & \phantom{-}1 & \phantom{-}0 \\
	\phantom{-}1/2 & \phantom{-}0 & -1
	\end{pmatrix} \begin{pmatrix}
	b_1 \\ b_2 \\ b_3
	\end{pmatrix} \\
	&= (a_1,a_2,a_3) \begin{pmatrix}
	b_1-b_2/2+b_3/2 \\
	-b_1/2+b_2 \\
	\phantom{-}b_1/2 - b_3
	\end{pmatrix} \\
	&= a_1(b_1-\frac{b_2}{2}+\frac{b_3}{2}) \\
	&\phantom{=}\hspace{6em}+ a_2(-\frac{b_1}{2}+b_2) \\
	&\phantom{=}\hspace{10em}+ a_3(\frac{b_1}{2} - b_3). 
	\end{align*}
	Ahora solo hace falta diagonalizar la matriz $S$ para obtener el polinomio diagonal buscado. Sea $v_1 = (1,0,0)$ y notemos que $\bform{v_1}{v_1}_S = 1$. El complemento ortogonal $v_1^\perp$ viene dado por
	\[ b_1-\frac{b_2}{2}+\frac{b_3}{2} = 0 \]
	que implica
	\[ v_1^\perp = \gen\left\{ b_2\begin{pmatrix}
	1/2 \\ 1 \\ 0
	\end{pmatrix} + b_3\begin{pmatrix}
	-1/2 \\ \phantom{-}0 \\ \phantom{-}1
	\end{pmatrix} \right\}. \]
	Notemos que $v_2=(1/2,1,0)$ es tal que $\bform{v_2}{v_2}_S = 3/4$. Y su complemento ortogonal $v_2^\perp$ viene dado por
	\[ \frac{3b_2}{4}+ \frac{b_3}{4} = 0\]
	y nuestro tercer vector $v_3$ es cualquier solución no trivial del sistema
	\[ \begin{cases}\displaystyle
	b_1-\frac{b_2}{2}+\frac{b_3}{2} = 0 \\[.5em] \displaystyle
	\frac{3b_2}{4}+ \frac{b_3}{4} = 0
	\end{cases} \]
	como por ejemplo $v_3 = (2,1,-3)$. Entonces la matriz 
	\[ P = \begin{pmatrix}
	1 & 1/2 & 2\\
	0 & 1 & 1\\
	0 & 0 & -3
	\end{pmatrix} \]
	es tal que
	\[ P^tSP = \begin{pmatrix}
	1 & 0 & 0 \\
	0 & 3/4 & 0 \\
	0 & 0 & -12 \\
	\end{pmatrix} \]
	
	Y finalmente nuestro polinomio $x^2-xy+y^2+xz-z^2$ es isométrico a $u^2+\frac{3}{4}v^2 -12 w^2$.
	
	\textsc{consideremos en segundo lugar} el polinomio $xy-xz-yz$. La matriz simétrica $S$ asociada a el es
	\[ S = \begin{pmatrix}
	0 & 1 & -1 \\
	1 & 0 & -1 \\
	-1 & -1 & 0
	\end{pmatrix} \]
	y la forma bilineal $\Bform_S$ viene dada,  para $v=(a_1,a_2,a_3)$ y $w=(b_1,b_2,b_3)$, por
	\begin{align*}
	v^tAw &= (a_1,a_2,a_3) \begin{pmatrix}
	0 & 1 & -1 \\
	1 & 0 & -1 \\
	-1 & -1 & 0
	\end{pmatrix} \begin{pmatrix}
	b_1 \\ b_2 \\ b_3
	\end{pmatrix} \\
	&= (a_1,a_2,a_3) \begin{pmatrix}
	b_2-b_3 \\
	b_1-b_3 \\
	-b_1-b_2
	\end{pmatrix} \\
	&= a_1(b_2-b_3) + a_2(b_1-b_3) + a_3(-b_1-b_2).
	\end{align*}
	Ahora solo hace falta diagonalizar la matriz $S$ para obtener el polinomio diagonal buscado. Sea $v_1 = (1,1,0)$ notemos que $\bform{v_1}{v_1}_S = 1$, y su complemento ortogonal $v_1^\perp$ esta dado por
	\[ b_1+b_2-2b_3 = 0 \]
	que implica
	\[ v_1^\perp = \gen\left\{ b_2\begin{pmatrix}
	-1 \\1 \\0 
	\end{pmatrix}  + b_3\begin{pmatrix}
	2 \\ 0 \\ 1
	\end{pmatrix}\right\}. \]
	Tomemos ahora $v_2 = (-1,1,0)$ y notemos que $\bform{v_2}{v_2}_S = -1$. Su complemento ortogonal $v_2^\perp$ esta dado por
	\[ b_1-b_2 = 0 \]
	Nuestro último vector $v_3$ es cualquier solución no trivial al sistema 
	\[ \begin{cases}
	b_1+b_2-2b_3 = 0 \\
	b_1-b_2 = 0
	\end{cases} \]
	como por ejemplo $v_3=(1,1,1)$. Y la matriz $P$ dada por
	\[ P = \begin{pmatrix}
	1 & -1 & 1\\
	1 & 1 & 1\\
	0 & 0 & 1\\
	\end{pmatrix} \]
	es tal que
	\[ P^tSP = \begin{pmatrix}
	2 & 0 & 0 \\
	0 & -2 & 0 \\
	0 & 0 & -2
	\end{pmatrix} \]
	y finalmente nuestro polinomio $xy-xz-yz$ es isométrico a $2v^2-2u^2-2w^2$.
	
	\textsc{consideremos en tercer lugar} el polinomio $xz+yw$. La matriz simétrica $S$ asociada a el es
	\[ S = \begin{pmatrix}
	0 & 0 & 1 & 0 \\
	0 & 0 & 0 & 1 \\
	1 & 0 & 0 & 0 \\
	0 & 1 & 0 & 0
	\end{pmatrix} \]
	y la forma bilineal $\Bform_S$ viene dada,  para $v=(a_1,a_2,a_3,a_4)$ y $w=(b_1,b_2,b_3,b_4)$, por
	\begin{align*}
	v^tAw &= (a_1,a_2,a_3,a_4)  \begin{pmatrix}
	0 & 0 & 1 & 0 \\
	0 & 0 & 0 & 1 \\
	1 & 0 & 0 & 0 \\
	0 & 1 & 0 & 0
	\end{pmatrix} \begin{pmatrix}
	b_1 \\ b_2 \\ b_3 \\ b_4
	\end{pmatrix} \\
	&= (a_1,a_2,a_3,a_4)  \begin{pmatrix}
	b_3 \\ b_4 \\ b_1\\ b_2 
	\end{pmatrix} \\
	&= a_1b_3 + a_2b_4 + a_3b_1 + a_4b_2.
	\end{align*}
	Ahora solo hace falta diagonalizar la matriz $S$ para obtener el polinomio diagonal buscado. Sea $v_1 = (1,0,1,0)$ notemos que $\bform{v_1}{v_1}_S = 1$, y su complemento ortogonal $v_1^\perp$ esta dado por
	\[ b_3+b_1 = 0 \]
	que implica
	\[ v_1^\perp = \gen \left\{ b_3\begin{pmatrix}
	-1 \\ 0 \\ 1 \\ 0
	\end{pmatrix}+ b_2\begin{pmatrix}
	0 \\ 1 \\ 0 \\ 0
	\end{pmatrix}+b_4\begin{pmatrix}
	0 \\ 0 \\ 0 \\ 1
	\end{pmatrix} \right\}. \]
	Tomemos ahora $v_2 = (-1,0,1,0)$ y notemos que $\bform{v_2}{v_2}_S = -1$. Su complemento ortogonal $v_2^\perp$ esta dado por
	\[ b_1-b_3 = 0. \]
	Ahora nuestor vector $v_3$ es cualqueir solución al sistema
	\[ \begin{cases}
	b_3+b_1 = 0 \\
	b_1-b_3 = 0
	\end{cases} \]
	como por ejemplo $v_3 = (0,1,0,1)$. Su complemento ortogonal $v_3^\perp$ esta dado por
	\[ b_4+b_2 = 0. \]
	El último vector $v_4$ es cualquier solución al sistema 
	\[ \begin{cases}
	b_3+b_1 = 0 \\
	b_1-b_3 = 0 \\
	b_4+b_2 = 0
	\end{cases} \]
	como $v_4 = (0,1,0,-1)$. Y la matriz $P$ dada por
	\[ P = \begin{pmatrix}
	1 & -1& 0& 0\\
	0 & 0& 1& 1\\
	1 & 1& 0& 0\\
	0 & 0& 1& -1\\
	\end{pmatrix} \]
	es tal que 
	\[ P^tAP =  \begin{pmatrix}
	2 & 0& 0& 0\\
	0 & -2& 0& 0\\
	0 & 0& 2& 0\\
	0 & 0& 0& -2\\
	\end{pmatrix} \]
	y nuestro polinomio $xz+yw$ es isométrico a $2t^2-2u^2+2v^2-2w^2$.

\section*{Ejercicio 5.3.5}
%Encontrar la forma cuadrática (expresada como un polinomio) asociada a cada una de las siguientes matrices simétricas:
%\[ S_1=\begin{pmatrix}
%2 & 4 \\ 4 & 1
%\end{pmatrix} \quad S_2=\begin{pmatrix}
%1 & 2 &1 \\
%1 & 0 & 3 \\
%1 & 3 & 2
%\end{pmatrix}\quad S_3=\begin{pmatrix}
%1 & 3 &4 \\
%3 & 0 & 1\\
%4 & 1 & 2
%\end{pmatrix} \]
	\begin{sol}
		Llamemos $S_1,S_2,S_3$ a cada una de las matrices del problema, numeradas en el orden que se leen (de izquierda a derecha). Para cada una de las matrices $S_i$ el polinomio cuadrático asociado a ellas se calcula como:
		\[ q_1 = v^tS_1v\quad q_2=v^tS_2v\quad q_3=v^tS_3v, \]
		donde $v$ es un vector de $\R^2$ en el caso de $q_1$ y un vector de $\R^3$ en el caso de $q_2$ y $q_3$.
		
		Entonces,
		\begin{align*}
		q_1 &=  v^tS_1v 
		= (x,y) \begin{pmatrix}
		2 & 4 \\ 4 & 1
		\end{pmatrix} \begin{pmatrix}
		x \\ y
		\end{pmatrix} \\
		&= (x,y) \begin{pmatrix}
		2x+4y \\
		4x+y
		\end{pmatrix} 
		= x(2x+4y) + y(4x+y)\\
		&= 2x^2+4xy+4yx+y^2 
		= 2x^2+8xy+y^2.
		\end{align*}
		Similarmente,
		\begin{align*}
		q_2 &= v^tS_2v 
		= (x,y,z) \begin{pmatrix}
		1 & 2 &1 \\
		1 & 0 & 3 \\
		1 & 3 & 2
		\end{pmatrix} \begin{pmatrix}
		x \\ y \\ z
		\end{pmatrix} \\
		&= (x,y,z)\begin{pmatrix}
		x+2y+z \\
		x+3z \\
		x+3y+2z
		\end{pmatrix} \\
		&=x(x+2y+z) + y(x+3z) + z(x+3y+2z) \\
		&=x^2+2xy+xz+yx+3yz+zx+3yz+2z^2 \\
		&=x^2+3xy+2xz+6yz+2z^2.
		\end{align*}
		Y finalmente,
		\begin{align*}
		q_3 &= v^tS_3v 
		= (x,y,z)\begin{pmatrix}
		1 & 3 &4 \\
		3 & 0 & 1\\
		4 & 1 & 2
		\end{pmatrix}\begin{pmatrix}
		x \\ y \\ z
		\end{pmatrix} \\
		&= (x,y,z)\begin{pmatrix}
		x+3y+4z \\
		3x+z\\
		4x+y+2z
		\end{pmatrix} \\
		&= x(x+3y+4z) + y(3x+z) + z(4x+y+2z) \\
		&= x^2+3xy+4xz + 3xy + yz + 4xz + yz + 2z^2 \\
		&= x^2 +6xy+8xz+2yz+2z^2.
		\end{align*}
	\end{sol}
\section*{Ejercicio 5.3.6}
%Sea $q$ la forma cuadrática en $\F^2$ dada por $XY$. Encuentre una transformación lineal invertible $T:\F^2\to\F^2$ tal que
%\[ q\big(T(a,b)\big) = a^2-b^2. \]
\begin{sol}
	Consideremos la transformación $T:\F^2\to\F^2$ dada, para todo $(a,b)\in\F^2$, por $T(a,b) = (a-b,a+b)$. Esta $T$ es lineal, pues si $a_1,b_1\in\F^2$ entonces
	\begin{align*}
		T(a+a_1,b+b_1) &= (a+a_1-b-b_1,a+a_1+b+b_1)\\
		 &= (a-b,a+b) + (a_1-b_1,a_1+b_1)\\
		 &= T(a,b) + T(a_1,b_1).
	\end{align*}
	
	También para esta $T$ se tiene que 
	\[ q\big(T(a,b)\big) = (a-b)(a+b) = a^2-b^2. \]
	
	Además la transformación $T$ es inyectiva, pues $\ker(T)=\{0\}$, y sobreyectiva, puesto que para cualquier vector $v$ en $\F^2$ se pueden hallar elementos $a,b\in\F$ tales que $v=(a-b,a+b)$. La biyectividad implica entonces que $T$ es invertible (más aún, su inversa $T\inv:\F^2\to\F^2$ esta dada por $T\inv(x,y) = 1/2(x+y,y-x)$).
	
	Luego la $T$ dada satisface las condiciones del problema.
\end{sol}
\section*{Ejercicio 5.3.8}
%Sea $Q:\V\to\F$ una forma cuadrática y $\Ve W$ un subespacio de $\V$.
%\begin{enumerate}
%	\item Demostrar que la restricción $Q|_{\Ve W}:\Ve W\to\F$ es una forma cuadrática.
%	\item Si $Q$ es no-degenerada, es $Q|_{\Ve W}$ no-degenerada?
%\end{enumerate}
\begin{sol} Veamos cada parte.
	\begin{enumerate}
		\item Sean $v,w\in\Ve W$ y $k\in\F$. Entonces
		\[ Q(kv) = k^2Q(v) \] 
		puesto que $v\in\V$ (por el hecho de que $\Ve W\subseteq\V$) y sabemos que $Q$ es una forma cuadrática sobre $\V$. Por una razón similar se tiene que
		\[ B_Q\bform{v}{w} = Q(v+w) - Q(v) - Q(w) \]
		es una forma bilineal simétrica.
		Y entonces $Q|_{\Ve W}$ es una forma cuadrática sobre $\Ve W$. 
	\end{enumerate}
\end{sol}
\section*{Ejercicio 5.4.4}
\begin{sol}
	Consideremos el determinante
	\[ \det(Q^tPQ) =  \det(P)\det(Q)^2. \]
	Y como $\det(P)\neq 0$ y $\det(Q)^2$ siempre es mayor que cero, se sigue que $\det(Q^tPQ)\neq 0$ y $Q^tPQ$ es no-singular.
\end{sol}
\section*{Ejercicio 5.4.5}
\begin{sol}
	Primero notemos que la bilinealidad y la simetría de $\Iprod{v}{w} = v^tSw$ se heredan directamente de la forma bilineal asociada a $S$ y de la simetría de $S$, respectivamente. Queda por demostrar la propiedad positiva definida.
	
	Como $S$ es simétrica y real sabemos que es congruente ortogonalmente con una matriz $D$ diagonal cuyos elementos de la diagonal son los valores propios, todos positivos, de $S$. Pero como la congruencia no afecta a la signatura de $S$ se sigue que la signatura de $D$ y la de $S$ son la misma. Como $D$ solo tiene elementos positivos en su diagonal su signatura es $n$, luego la signatura de $S$ también es $n$ y $S$ es positiva definida.
\end{sol}
\section*{Ejercicio 5.4.6}
\begin{sol}
	Como $P$ es positiva definida, simétrica y real existe una matriz ortogonal $B$ tal que $P=B^tDB$ y $D$ es una matriz diagonal cuyos elementos de la diagonal (los valores propios de $P$) son todos positivos.
	
	Consideremos el determinante
	\[ \det(P) = \det(B^tDB) = \det(D)\det(B)^2. \]
	Como $\det(D)$ es el producto de la diagonal y $\det(B)^2$ es siempre mayor que cero, se sigue que $\det(P)>0$.
	
	Si $A,B\in\Ve M_{n\t n}(\R)$ entonces $\traz(AB) = \traz(BA)$. De esto se sigue que
	\[ \traz(P) = \traz(B^tDB) = \traz(D) \]
	y como $\traz(D)>0$ se sigue que $\traz(P)>0$.
\end{sol}
\section*{Ejercicio 5.4.7}
\begin{sol}
	La siguiente matriz $C=\left( \begin{smallmatrix}
	1& -2 & 1 & 2 \\
	0& 1& -2& -3 \\
	0& 0& 1& 0\\
	0& 0& 0& 1\\
	\end{smallmatrix} \right)$,
	diagonaliza a la matriz $A$:
	\[ C^tAC = \begin{pmatrix}
	-1 & 0 & 0 & 0 \\
	0 & 1 & 0 & 0 \\
	0 & 0& 0& 0 \\ 
	0 & 0& 0& 0 \\
	\end{pmatrix} \]
	Entonces el rango de $A$ es $2$ y la signatura de $A$ es $0$.
\end{sol}
\end{document}