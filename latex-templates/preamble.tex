% So that TeX doesn't complain about small
% underfull or overfull boxes
\hfuzz1pc
% Make the overfull marker bigger
\overfullrule=2cm

% Font setup.
\usepackage{cfr-lm}
\usepackage[T1]{fontenc}
% 20% bigger line height
\linespread{1.05}
% Don't put extra space after periods
\frenchspacing

\KOMAoptions{
    paper = letter,
    BCOR = 0mm,
    twoside = false,
    fontsize = {10},
    DIV = calc,
}

% Make bibliography more compact, no indents.
\KOMAoption{toc}{flat}

% Language support, usually changes between english
% and spanish.
\usepackage[spanish,es-noindentfirst]{babel}
%\usepackage[english]{babel}
\usepackage{csquotes}

% Bibliography
\usepackage[
    backend=biber,
    style=numeric-comp,
    backref=true,
    backrefstyle=two,
    abbreviate=true
]{biblatex}
\addbibresource{~/git/Misc-LaTeX-files/bib/general.bib}
\addbibresource{~/git/Misc-LaTeX-files/bib/math-books.bib}

% Graphics, mainly to insert images or
% single page PDFs.
\usepackage{graphicx}
\usepackage[dvipsnames]{xcolor}
% Handy command to typeset URLs
\usepackage{hyperref}
\hypersetup{
    colorlinks=true,
    linkcolor=Mahogany,
    filecolor=Mahogany,
    urlcolor=Black,
    citecolor=Mahogany,
}
\usepackage{url}
%\urlstyle{same}
\usepackage{metalogo}

\usepackage{minted}

% Font style and size for title
\setkomafont{title}{\fontseries{m}}
% Font style for the subject
\setkomafont{subject}{\normalfont\bfseries}
% Font style for subtitle
\setkomafont{subtitle}{\normalfont\itshape}
\setkomafont{author}{\large}
\setkomafont{date}{\normalsize}
\setkomafont{section}{\fontseries{m}\Large}
\setkomafont{subsection}{\fontseries{m}\large}
\setkomafont{subsubsection}{\fontseries{m}\normalsize}
% Footnotes
\deffootnote{2.0em}{1.5em}{\thefootnotemark.\ }

% CUSTOM MACROS
% ToC not in sans-serif
% \let\oldtoc\tableofcontents
% \renewcommand{\tableofcontents}{%
%     {
%         \renewcommand{\sffamily}{\rmfamily}
%         \oldtoc%
%     }
% }
% math macros
\renewcommand{\Rn}{\mathbb{R}^{\mathrm{n}}}
\newcommand{\Rm}{\mathbb{R}^{\mathrm{m}}}
\newcommand{\R}{\mathbb{R}}
\newcommand{\N}{\mathbb{N}}
\newcommand{\devpart}[2]{\frac{\partial  #1}{\partial #2}}
\renewcommand{\vec}[1]{\mathbf{#1}}
\newcommand{\norm}[1]{\left\lvert #1 \right\rvert}
\DeclareMathOperator{\img}{img}
\DeclareMathOperator{\gen}{span}
