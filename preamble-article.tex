% This TeX file is auto generated from a Org file.
% As such, is not properly documented.
% Please refer to the Org or HTML file instead for explanations.

\hfuzz1pc
\overfullrule=2cm

\usepackage[spanish,es-noindentfirst]{babel}
\usepackage{csquotes}

\usepackage[
	includehead,
	includefoot,
	letterpaper,
	top=2cm,
	bottom=2cm,
	left=4.5cm,
	right=4.5cm,
	marginparsep=0.5cm,
	marginparwidth=4cm,
]{geometry}

\usepackage{mathtools}
\DeclareMathOperator{\Rea}{Re}
\DeclareMathOperator{\Ima}{Im}
\DeclareMathOperator{\car}{car}
\DeclareMathOperator{\traz}{tr}
\DeclareMathOperator{\gen}{gen}
\DeclareMathOperator{\mcm}{mcm}

\usepackage[utf8]{inputenc}
\usepackage[T1]{fontenc}

\usepackage{ccfonts}
\usepackage{sourcesanspro}
\usepackage{GoMono}
\frenchspacing

\usepackage{eulervm}

\usepackage[final]{microtype}

\PassOptionsToPackage{final}{graphicx}
\PassOptionsToPackage{dvipsnames}{xcolor}

\usepackage{%
	xcolor,%
	graphicx,%
	cancel,%
	booktabs,
	hyphenat,
	authoraftertitle,
	pdfpages,
	metalogo
}

\usepackage[
	backend=biber,
	backref=true,
	citestyle=authoryear-comp,
	style=chicago-authordate ,
	sorting=ynt
]{biblatex}
\addbibresource{/home/jhonny/git/Misc-LaTeX-files/bib/general.bib}

\usepackage{url} 
\usepackage{hyperref} 
\hypersetup{
colorlinks=true,
linkcolor=NavyBlue,
urlcolor=NavyBlue,
citecolor=NavyBlue
}
\usepackage[spanish,nameinlink]{cleveref}

\newcommand{\marfont}{}

\renewcommand*{\marginfont}{\marfont}
\let\oldmarginpar\marginpar
\renewcommand{\marginpar}[1]{
	\oldmarginpar{\raggedright\marfont #1}
}

\newcounter{nota}
\newcommand{\nota}[1]{\refstepcounter{nota}\textsuperscript{\thenota}%
	\marginpar{%
		\raggedright\itshape\thenota. #1%
	}
}

\renewcommand{\footnote}{\nota}

\usepackage{enumitem} 
\setlist[enumerate]{left=-11pt,nosep}
\setlist[description]{font=\normalfont,leftmargin=\parindent}
\setlist[itemize]{label={\small\textbullet},left=-11pt}

\newcommand{\Asignatura}{}
\newcommand{\asignatura}[1]{\renewcommand{\Asignatura}{#1}}

\makeatletter
\def\@maketitle{%
  \newpage
  \null
  \let \footnote \thanks
  \begin{flushleft}
	  {\MakeUppercase{\@title}
	  	\marginnote{
	  		\parbox{3cm}{\raggedright\normalfont\normalsize\@author}
  		}[.5em]\par
  	  }
	\smallskip
	  {\Asignatura\ (\today)\par}
  \end{flushleft}
  \vskip 1.5\baselineskip
}
\makeatother

\usepackage[final]{listings} 
\lstset{
language=R, 
numbers=left, numberstyle=\tiny\ttfamily, stepnumber=2, numbersep=5pt, 
basicstyle=\ttfamily, 
stringstyle=\ttfamily,
commentstyle=\itshape,
breaklines=true,
postbreak=\mbox{$\hookrightarrow$\enspace},
columns=flexible
}

\usepackage{caption} 
\captionsetup{
font={rm},
justification=raggedright,
singlelinecheck=false,
skip=3pt
}

\usepackage[explicit]{titlesec}

\titleformat{\section}[hang]
{\flushleft\scshape}
	{\thesection}
	{1em}
	{#1}
	[]
\titlespacing*{\section}
	{0em}
	{1.5\baselineskip}
	{1.5\baselineskip}
\titleformat{\subsection}
{\flushleft\itshape}
	{\thesubsection}
	{.5em}
	{#1}
	[]
\titlespacing*{\subsection}
	{0em}
	{1\baselineskip}
	{1\baselineskip}
\titleformat{\paragraph}[runin]
	{\bfseries}
	{}
	{0em}
	{\MakeLowercase{#1}}
	[.]
\titlespacing*{\paragraph}
	{0em}
	{1\baselineskip}
	{5pt}

\let\oldtoc\tableofcontents
\renewcommand{\tableofcontents}{\marginpar{\bigskip\oldtoc}}
\usepackage{titletoc}

\titlecontents{section}
	[0em]
	{\vspace{-10pt}}
	{\contentsmargin{0pt}}
	{\contentsmargin{0pt}}
	{\contentspage}
	[\vspace{15pt}]
\titlecontents{subsection}
	[.5em]                              
	{\vspace{-11pt}}
	{\contentsmargin{0pt}\footnotesize}
	{\contentsmargin{0pt}\footnotesize}        
	{\contentspage}                 
	[\vspace{14pt}]

\usepackage{fancyhdr}

\renewcommand{\headrulewidth}{0pt}
\setlength{\headheight}{14pt}

\pagestyle{fancy}
\fancyhf{}
\fancyhead[L]{\ifodd\value{page}\MyTitle\else\Asignatura\fi}
\fancyhead[R]{\thepage}
\fancypagestyle{plain}{%
	\fancyhead[R]{}
	\fancyhead[L]{}
	\fancyfoot[R]{}%
	\fancyfoot[L]{}
	\fancyfoot[C]{}
}

\usepackage[thmmarks]{ntheorem}
	\theoremstyle{plain}
	\theoremindent0cm
	\theorempreskip{0cm}
	\theorempostskip{0cm}
	\theoremheaderfont{\hspace*{\parindent}\upshape}
	\theorembodyfont{\itshape}
	\theoremseparator{.}
	\newtheorem{teo}{Teorema}[section]
	\newtheorem{cor}{Corolario}[teo]
	\newtheorem{prop}{Proposición}[section]
	\newtheorem{lem}{Lema}[section]
	\theoremstyle{nonumberplain}
	\theoremheaderfont{\normalfont}
	\theorembodyfont{\upshape}
	\newtheorem{proof}{Demostración}
	\theoremstyle{plain}
	\theorempreskip{1em}
	\theorempostskip{1em}
	\theoremheaderfont{\upshape}
	\theorempostwork{\noindent}
	\newtheorem{definition}{Definición}[section]
