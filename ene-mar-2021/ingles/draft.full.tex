\documentclass{scrartcl}
% So that TeX doesn't complain about small
% underfull or overfull boxes
\hfuzz1pc
% Make the overfull marker bigger
\overfullrule=2cm

% Font setup.
\usepackage[T1]{fontenc}
\usepackage{tgtermes,zlmtt}
\usepackage[scale=.85]{tgheros}
\usepackage[notext]{stix}
% 20% bigger line height
\linespread{1.02}
% Don't put extra space after periods
\frenchspacing

\KOMAoptions{
paper = letter,
BCOR = 0mm,
twoside = false,
pagesize = luatex,
fontsize = {11},
DIV = calc,
}

% Make bibliography more compact, no indents.
\KOMAoption{toc}{flat}

% Language support, usually changes between english
% and spanish.
%\usepackage[spanish,es-noindentfirst]{babel}
\usepackage[english]{babel}

% Graphics, mainly to insert images or
% single page PDFs.
\usepackage{graphicx}
\usepackage[dvipsnames]{xcolor}
% Handy command to typeset URLs
\usepackage[dvips]{hyperref}
\hypersetup{
    colorlinks=true,
    linkcolor=Mahogany,
    filecolor=Mahogany,
    urlcolor=Mahogany,
    citecolor=Mahogany,
}
\usepackage{url}
\urlstyle{same}


% Font style and size for title
\setkomafont{title}{\normalfont}
% Font style for the subject
\setkomafont{subject}{\normalfont}
% Font style for subtitle
\setkomafont{subtitle}{\normalfont\itshape}
\setkomafont{section}{\normalfont\Large}
\setkomafont{subsection}{\normalfont\large}
\setkomafont{subsubsection}{\normalfont}

% CUSTOM MACROS
% ToC not in sans-serif
\let\oldtoc\tableofcontents
\renewcommand{\tableofcontents}{{\renewcommand{\sffamily}{\rmfamily}\oldtoc}}
% Macros for writing code
\newcommand{\icode}[1]{\textsf{#1}}
%
\begin{document}
%
%
\title{Draft for math class}
\subtitle{Main talking points for my class about subtraction}
\subject{ID2131}
\titlehead{Universidad Simón Bolívar\hfill Caracas, Venezuela}
\author{by \\ Jhonny Lanzuisi}
\date{\today}
\maketitle

\begin{abstract}
  This draft is a short version of the class,
  so that I can remember what to say.
  For more detail, check the slides.
\end{abstract}

\section{Opening}
\label{sec:open}

Hi, I'm Jhonny.

This class'll be about the mathematical operation
known as \emph{subtraction}.
The goal is to know what this operation does and
how to do it yourself.

\section{Explanation}
\label{sec:expla}

Subtraction is an operation performed between two numbers.
It is not commutative, but it is associative, as a 'zero',
elements have an inverse.

It is a binary operation.
If there are more than two numbers,
associativity is used.

These two numbers we subtract are, generally,
of two types (make reference to complex numbers here)
whole numbers (including negatives) and
real numbers (decimals).

\section{Examples}
\label{sec:examp}

\begin{itemize}
\item Simple subtraction with whole numbers.
\item Same as above but explaining what happens with \(-(-a)\).
\item Simple example with reals.
\item A longer subtraction.
\item Combining reals and integers.
\end{itemize}

\section{Closing}
\label{sec:close}

Now you should have an idea of what subtraction is,
how to do it,
and some interesting properties it has.

This is all for today, good bye.
\newpage
\section*{Colophon \& Copyright}
This document was typeset using \TeX\footnote{\TeX\ is
a typesetting software, free and open source,
developed by Donald Knuth. \LaTeX\ is available in
all major operating systems.}
and the \LaTeXe\ macros in a GNU+Li\-nux operating system.
The editor used for editing the text was Emacs\footnote{The main version
of Emacs today is the one by GNU. GNU Emacs comes from 1985,
but it's a direct descendent of an older Emacs.
Emacs means \textit{E}dit \textit{MAC}ro\textit{S}.}.
The main typefaces used are Times Roman,
Helvetica and Latin Modern Typewritter, for text and for maths
the STIX fonts\footnote{In \LaTeX\ this amounts to using the packages:
\icode{tgtermes},\icode{tgheros},\icode{zlmtt} and \icode{stix}.}.
\medskip
\begin{quote}\footnotesize
  \begin{quote}
  Copyright (C)  2021 Jhonny Lanzuisi.\\
  Permission is granted to copy, distribute and/or modify this document
  under the terms of the GNU Free Documentation License, Version 1.3
  or any later version published by the Free Software Foundation;
  with no Invariant Sections, no Front-Cover Texts, and no Back-Cover Texts.
  A copy of the license can be found at\\
  \url{https://www.gnu.org/licenses/fdl-1.3.en.html}.\par
  \end{quote}
\end{quote}
\newpage
\tableofcontents
\end{document}