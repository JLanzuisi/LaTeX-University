%
\title{Draft for math class}
\subtitle{Main talking points for my class about subtraction}
\subject{ID2131}
\titlehead{Universidad Simón Bolívar\hfill Caracas, Venezuela}
\author{by \\ Jhonny Lanzuisi}
\date{\today}
\maketitle

\begin{abstract}
  This draft is a short version of the class,
  so that I can remember what to say.
  For more detail, check the slides.
\end{abstract}

\section{Opening}
\label{sec:open}

Hi, I'm Jhonny.

This class'll be about the mathematical operation
known as \emph{subtraction}.
The goal is to know what this operation does and
how to do it yourself.

\section{Explanation}
\label{sec:expla}

Subtraction is an operation performed between two numbers.
It is not commutative, nor associative, it has a 'zero',
elements have an inverse.

It is a binary operation.
If there are more than two numbers,
associativity is used.

These two numbers we subtract are, generally,
of two types (make reference to complex numbers here)
whole numbers (including negatives) and
real numbers (decimals).

\section{Examples}
\label{sec:examp}

\begin{itemize}
\item Simple subtraction with whole numbers.
\item Same as above but explaining what happens with \(-(-a)\).
\item Simple example with reals.
\item A longer subtraction.
\item Combining reals and integers.
\end{itemize}

\section{Closing}
\label{sec:close}

Now you should have an idea of what subtraction is,
how to do it,
and some interesting properties it has.

This is all for today, good bye.
