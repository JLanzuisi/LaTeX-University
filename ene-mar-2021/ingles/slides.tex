\documentclass[landscape]{slides}
\usepackage[landscape]{geometry}

% Font setup.
\usepackage[T1]{fontenc}
\usepackage{tgtermes}
\usepackage[scale=.85]{tgheros}
\usepackage{zlmtt}
\usepackage[notext]{stix}

\usepackage{url}
\usepackage[dvipsnames]{xcolor}

\renewcommand{\sffamily}{\rmfamily}

\newcommand{\colt}[1]{\textcolor{MidnightBlue}{\textbf{#1}}}
\newcommand{\sldt}[1]{\colt{\large #1}}

\begin{document}
\raggedright
{\small Universidad Simón Bolívar}
\bigskip

{\small ID2131-English for mathematicians I}\\[.5em]
\colt{\Large Class about subtraction}

{\LARGE
\[ \alpha - \beta = \gamma \]
}
\vfill
by\\
Jhonny Lanzuisi

{\small
  \today\\
Final evaluation
Jan-Mar 2021}

\begin{slide}
  \sldt{What is it?}

  An operation between two numbers.
  Something we \emph{do} to said numbers.

  {\Large
    \[ a \;\textcolor{red}{-}\; b = c\]
  }
\end{slide}

\begin{slide}
  \sldt{What is it?}

  An operation between two numbers.
  Something we \emph{do} to said numbers.

  {\Large
    \[ \textcolor{blue}{a} - \textcolor{red}{b} = \textcolor{green}{c} \]
  }
\end{slide}

\begin{slide}
  \sldt{What is it?}

  An operation between two numbers.
  Something we \emph{do} to said numbers.
  That something is \emph{taking away}, i.e.,
  we take away 7 units from 10.

  {\Large
    \[ \textcolor{Gray}{a - b = c} \]
    \[ 10 - 7 = 3 \]
  }
\end{slide}

\begin{slide}
  \sldt{Visualization}

  Imagine the \emph{real line}.

  \setlength{\unitlength}{2cm}
  \begin{picture}(10,2)
    \thicklines
    \put(0,1){\line(1,0){10}}
    \put(5,0.75){\line(0,1){0.5}}
    \put(5,1.30){0}
    \put(6,0.75){\line(0,1){0.5}}
    \put(6,1.30){1}
    \put(7,0.75){\line(0,1){0.5}}
    \put(7,1.30){2}
    \put(8,0.75){\line(0,1){0.5}}
    \put(8,1.30){3}
    \put(9,0.75){\line(0,1){0.5}}
    \put(9,1.30){4}
    \put(0,1){\line(1,0){10}}
    \put(4,0.75){\line(0,1){0.5}}
    \put(4,1.30){-1}
    \put(3,0.75){\line(0,1){0.5}}
    \put(3,1.30){-2}
    \put(2,0.75){\line(0,1){0.5}}
    \put(2,1.30){-3}
    \put(1,0.75){\line(0,1){0.5}}
    \put(1,1.30){-4}
    \put(5.5,0.5){\vector(0,1){0.5}}
    \put(5.5,0.2){$\frac{1}{2}$}
    \put(8.14,0.5){\vector(0,1){0.5}}
    \put(8.14,0.3){$\pi$}
    \put(8.14,0){\footnotesize $3.141592\dots$}
  \end{picture}
\end{slide}

\begin{slide}
  \sldt{Visualization}

  Imagine the \emph{real line}. Then subtraction means \emph{moving to the left}.

  \setlength{\unitlength}{2cm}
  \begin{picture}(10,2)
    \thicklines
    \put(0,1){\line(1,0){10}}
    \put(5,0.75){\line(0,1){0.5}}
    \put(5,1.30){0}
    \put(6,0.75){\line(0,1){0.5}}
    \put(6,1.30){\textcolor{green}{1}}
    \put(7,0.75){\line(0,1){0.5}}
    \put(7,1.30){2}
    \put(8,0.75){\line(0,1){0.5}}
    \put(8,1.30){\textcolor{red}{3}}
    \put(9,0.75){\line(0,1){0.5}}
    \put(9,1.30){\textcolor{blue}{4}}
    \put(0,1){\line(1,0){10}}
    \put(4,0.75){\line(0,1){0.5}}
    \put(4,1.30){-1}
    \put(3,0.75){\line(0,1){0.5}}
    \put(3,1.30){-2}
    \put(2,0.75){\line(0,1){0.5}}
    \put(2,1.30){-3}
    \put(1,0.75){\line(0,1){0.5}}
    \put(1,1.30){-4}
    \put(9,1.8){\vector(-1,0){3}}
  \end{picture}
\end{slide}

\begin{slide}
  \sldt{Properties}

  Changing the order \emph{changes} the result.

  {\Large
    \[ a - b \;\textcolor{blue}{\neq}\; b - a \]
  }

  \vfill
  \hfill
  (Non-commutativity)
\end{slide}

\begin{slide}
  \sldt{Properties}

  Changing the order \emph{changes} the result.

  {\Large
    \[ \textcolor{Gray}{a - b \neq b - a} \]
    \[ \underbrace{10 - 7}_3 \neq \underbrace{7 - 10}_{-3} \]
  }

  \vfill
  \hfill
  (Non-commutativity)
\end{slide}

\begin{slide}
  \sldt{Properties}

  Changing the order \emph{changes} the result.
  But only for a \emph{negative sing}.
  If we get rid of it (absolute value)
  then we do get the same amount.

  {\Large
    \[ \textcolor{Gray}{a - b \neq b - a} \]
    \[ \underbrace{10 - 7}_3 \neq \underbrace{7 - 10}_{-3} \]
  }

  \vfill
  \hfill
  (Non-commutativity)
\end{slide}

\begin{slide}
  \sldt{Properties}

  Grouping \emph{also changes} the result.

  {\Large
    \[ (a-b) - c \neq a - (b-c)\]
  }

  \vfill
 \hfill
 (Non associative)
\end{slide}

\begin{slide}
  \sldt{Properties}

  Grouping \emph{also changes} the result.

  {\Large
    \textcolor{Gray}{\[ (a-b) - c \neq a - (b-c)\]}
    \[(10 - 7) - 2 \neq 10 - (7-2)\]
  }

  \vfill
 \hfill
 (Non associative)
\end{slide}

\begin{slide}
  \sldt{Properties}

  Grouping \emph{also changes} the result.
  {\Large
    \textcolor{Gray}{\[ (a-b) - c \neq a - (b-c)\]}
   \[\overbrace{\underbrace{(10 - 7)}_3 - 2}^1 \neq \overbrace{10 - \underbrace{(7-2)}_5}^5\]
 }

 \vfill
 \hfill
 (Non associative)
\end{slide}

\begin{slide}
  \sldt{Properties}

  There is a number, which we call `0',
  that has the property of \emph{subtracting nothing}.
  This number 0 is unique.

  {\Large
    \[ a - 0 = a\]
  }

  \vfill\hfill
  (Neutral element)
\end{slide}

\begin{slide}
  \sldt{Properties}

  Given any number,
  there is another number such
  that their subtraction is zero.
  This other number (\(b\)) is unique.

  {\Large
    \[ a - b = 0\]
  }

  This number \(b\) is, of course,
  the same number \(a\):
  {\Large
    \[ 2 - 2 = 0\]
  }
  \hfill (Inverses)
\end{slide}

\begin{slide}
  \sldt{Examples and practice}
  \newcommand{\qmk}{\textcolor{blue}{?}}
  \begin{eqnarray*}
    & 9-8 = \qmk \\
    & 3-8 = \qmk \\
    & 3-(-4) = \qmk \\
    & (-2)-7 = \qmk \\
    & \frac12 - \frac34 = \qmk\\
    & \pi - (\frac{3\pi}{2} - \frac\pi2) = \qmk
  \end{eqnarray*}
\end{slide}

\begin{slide}
  \sldt{Examples and practice (solutions)}
  \newcommand{\qmk}{\textcolor{blue}{?}}
  \begin{eqnarray*}
    & 9-8 = 1\\
    & 3-8 = -5 \\
    & 3-(-4) = 7 \\
    & (-2)-7 = -9 \\
    & \frac12 - \frac34 = -\frac14\\
    & \pi - (\frac{3\pi}{2} - \frac\pi2) = 0
  \end{eqnarray*}
\end{slide}

\begin{slide}
\sldt{License and colophon}

These slides were made with \LaTeXe\
and the slides document class
on a GNU/Linux system.
The typeface used is Times Roman.

\begin{quote}\tiny
  \begin{quote}
  Copyright (C)  2021 Jhonny Lanzuisi.\\
  Permission is granted to copy, distribute and/or modify this document
  under the terms of the GNU Free Documentation License, Version 1.3
  or any later version published by the Free Software Foundation;
  with no Invariant Sections, no Front-Cover Texts, and no Back-Cover Texts.
  A copy of the license can be found at
  \url{https://www.gnu.org/licenses/fdl-1.3.en.html}.\par
  \end{quote}
\end{quote}
\end{slide}
\end{document}