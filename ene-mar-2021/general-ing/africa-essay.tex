%
\title{Essay questions on the unit on Africa}
\subtitle{Based on the readings of selected works
          by the following African authors: Nadine Gordimer,
          Wole Soyinka, Chinua Achebe.
}
\titlehead{Universidad Simón Bolívar\hfill Caracas, Venezuela}
\subject{Literatura en ingles}
\author{by \\ Jhonny Lanzuisi}
\date{\today}
\maketitle

\section[Second Question]{In {\normalfont\em what were you dreaming}, what do you learn about the system of Apartheid and how it affected the people?}

South African author Nadine Gordimier,
recipient of the 1991 Nobel Prize in Literature\cite{noauthor_nadine_2021},
describes trough her short story \textit{what were you dreaming}
the never ending pain of Apartheid.
In this text  we see numerous aspects
of South African Apartheid,
among which the following the following
are worth noting:
\begin{itemize}
\item Systemic racism vertebrated by differences in skin colour and
      and European ascendancy. The different tribes born around this system.
\item Forced removals and other consequences of the homeland system \cite{noauthor_apartheid_2021}.
\item The overall situation of poverty that black people endured.
\item The social standing of women in South Africa.
\item The existence of the white anti-apartheid movement.
\end{itemize}

Some of these issues will be more thoroughly examined in the following.

We know that South African Apartheid is not the only
racist system in history. Many have existed and discriminated
against different races.
But the Apartheid is relevant because it ended relatively
recently, in 1991 \cite{noauthor_apartheid_2021}.

The inner workings of said system are not entirely revealed to us
in the work by Gordimer, but the overall social stratification is.
This society was divided among four groups: whites, Africans, Indian,
and coloureds.

In the story we're treated to a colored man as one of the protagonists,
and the way he speaks of \textit{the blacks}:
\begin{quote}
  And the blacks---when they stop for you they ask for money.
  The want you must pay, like for a taxi! The blacks! \cite[Page 213--214]{nadine_dreaming}
\end{quote}
gives the first hint at how this society was divided.
This man, even though he's almost a black man himself,
sees himself as something different.

This man tells about his family, all broken up, poor and sick,
and in this telling he's in a way also speaking
about most of his brothers and sisters.
The truth is that unless one was very lucky,
or white,
living in South Africa was a living tragedy.

Later on another one of the protagonists,
a white women that's against Apartheid,
starts to describe the horrible conditions in which
so many Africans are forced to live.

All of this gives us an idea of the state so many South Africans lived in.
The fact that this is described so accurately and so heartbreakingly
by Gordimier is, of course, no surprise. She was an active anti-Apartheid
protester and writer, as well as advisor to president Nelson Mandela \cite[Second paragraph]{noauthor_nadine_2021}.

One aspect that's present throughout the whole story
is the fact that people were treated as disposable goods,
they were something to keep alive just enough to work,
they were objects:
\begin{quote}
  people as movable goods. People packed onto trucks (\dots)
  People dumped somewhere else. People as figures. \cite[Page 219]{nadine_dreaming}
\end{quote}

This is, of course, a consequence of a racist inhuman government.
But also reveals the existence of a complacent population,
and a tolerating international sphere,
that could not condemn the horrors.

The very place the whole story takes place in,
a car owned by a white man,
gives a good idea of the level of segregation.
This is because the coloured protagonist
can't believe he has been given the right to accompany
white people in their car.

We have to keep in mind that Apartheid society was fully segregated,
whites and blacks lived in different areas, used different public transports,
went to different hospitals, worked in different areas, etc.

One of the consequences of such a drastic system was that,
as the colored man does,
people get used to and internalize the situation.
He's a human being and therefore has inherent dignity,
this he surely knows,
but he (and every other non white) speaks
of the white protagonist as \textit{masters}.

This is of course a consequence of trying
to keep colonial rule and following on the
footsteps of slave owners.
Black people had to think of their situation as irreversible,
and of white people as their natural superiors.

The female of the three protagonists,
a white old women,
gives a glimpse of the live of women under Apartheid.
On one hand there are the non whites,
who have to endure misogyny on top of poverty.
And then the white women, that's prohibited to engage with
or even to think of a black man in any way other than a slave.

Lastly, this same women,
gives a hint of another South Africa.
She's not suspicious of the coloured man,
neither is she suspicious about other black men or women.
She's not a racist. What's even more,
she perfectly understands the situation in which
millions of South Africans live.
This is the base upon which support for the end of Apartheid
would begin in the white medium/upper class.
%
\bibliographystyle{ieeetr}
\bibliography{/home/jhonny/git/Misc-LaTeX-files/bib/general.bib}
\addcontentsline{toc}{section}{References}
