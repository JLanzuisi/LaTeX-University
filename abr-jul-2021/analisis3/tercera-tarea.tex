\documentclass{scrartcl}

% So that TeX doesn't complain about small
% underfull or overfull boxes
\hfuzz1pc
% Make the overfull marker bigger
\overfullrule=2cm

% Font setup.
\usepackage[neoeuler]{fontsetup}
\defaultfontfeatures{Ligatures=TeX}
\newfontfamily{\light}{League Spartan Light}
% 20% bigger line height
\linespread{1.04}
% Don't put extra space after periods
\frenchspacing

\KOMAoptions{
    paper = letter,
    BCOR = 0mm,
    twoside = false,
    fontsize = {10},
    DIV = calc,
}

% Make bibliography more compact, no indents.
\KOMAoption{toc}{flat}

% Language support, usually changes between english
% and spanish.
%\usepackage[spanish,es-noindentfirst]{babel}
\usepackage[english]{babel}
\usepackage{csquotes}

% Bibliography
\usepackage[
    backend=biber,
    style=numeric-comp,
    backref=true,
    backrefstyle=two,
    abbreviate=true
]{biblatex}
\addbibresource{~/git/Misc-LaTeX-files/bib/general.bib}
\addbibresource{~/git/Misc-LaTeX-files/bib/math-books.bib}

% Graphics, mainly to insert images or
% single page PDFs.
\usepackage{graphicx}
\usepackage[dvipsnames]{xcolor}
% Handy command to typeset URLs
\usepackage{hyperref}
\hypersetup{
    colorlinks=true,
    linkcolor=Mahogany,
    filecolor=Mahogany,
    urlcolor=Black,
    citecolor=Mahogany,
}
\usepackage{url}
%\urlstyle{same}
\usepackage{metalogo}


% Font style and size for title
\setkomafont{title}{\fontseries{m}\light}
% Font style for the subject
\setkomafont{subject}{\normalfont}
% Font style for subtitle
\setkomafont{subtitle}{\normalfont}
\setkomafont{author}{\large}
\setkomafont{date}{\normalsize}
\setkomafont{section}{\fontseries{m}\light\Large}
\setkomafont{subsection}{\fontseries{m}\light\large}
\setkomafont{subsubsection}{\fontseries{m}\normalfont}
% Footnotes
\deffootnote{2.0em}{1.5em}{\thefootnotemark.\ }

% CUSTOM MACROS
% ToC not in sans-serif
\let\oldtoc\tableofcontents
\renewcommand{\tableofcontents}{%
    {
        \renewcommand{\sffamily}{\rmfamily}
        \oldtoc%
    }
}
% Macros for writing code
\newcommand{\icode}[1]{\verb~#1~}
% Urls
% \let\oldurl\url
% \newcommand{\umark}{\text{\textcolor{Mahogany}{\texttt{o}}}}
% \renewcommand{\url}[1]{\oldurl{#1}\(^{\umark}\)}
% emph to bold
\renewcommand{\emph}{\textbf}
% math macros
\renewcommand{\Rn}{\mathbb{R}^{\mathrm{n}}}
\newcommand{\Rm}{\mathbb{R}^{\mathrm{m}}}
\newcommand{\R}{\mathbb{R}}
\newcommand{\N}{\mathbb{N}}

\begin{document}

%
\title{Tercera tarea}
\subtitle{Tercera evaluación del curso}
\subject{Análisis III}
\titlehead{Universidad Simón Bolívar\hfill Caracas, Venezuela}
\author{{\normalsize hecho por} \\ Jhonny Lanzuisi}
\date{\today}
\maketitle
\thanks{Una preuba}

\exercise
%
Sea \(f\colon U\subseteq\R^{2}\to\R\) continua en el abierto \(U\)
tal que
\[
(x^2+y^2)f(x,y) + f(x,y)^3 = 1
\]
para cualquier \((x,y)\in U\). Demuestre que \(f\in C^\infty (U)\).

\solution
%
La función \(z = f(x,y)\) cumple la relación
\[
(x^2+y^2)z + z^3 = 1,
\]
por lo que no puede ser 0 nunca, de lo contrario tendríamos
\(0=1\).

Por el teorema de la función implícita \(f\) debe ser
de clase \(C^1\) y,
\[
\devp{z}{x} = \frac{2xz}{x^2+y^2+3z^2},
\devp{z}{y} = \frac{2yz}{x^2+y^2+3z^2}.
\]
Los denominadores nunca se hacen cero puesto que
\(z\) nunca es cero. Esto significa que podemos
repetir el proceso anterior para obtener las derivadas segundas,
que serán nuevamente continuas. Siguiendo de esta forma se obtiene
que \(f\in C^\infty(U)\).

\exercise
%
Determine los puntos críticos de \(f(x,y) = \iprod{x}{y}\),
\(f\colon\R^{2m}\to\R\), restricta a la esfera unitaria
\(\norm{x}^2+\norm{y}^2 = 1\) y muestre de aquí la desigualdad
de Cauchy-Schwarz.

\solution
%
Utilizando el método de los multiplicadores de Lagrange,
hallar los puntos críticos de \(f\) se reduce a resolver
el siguiente sistema de ecuaciones:
\[
\Delta f(x,y) = \lambda\Delta g(x,y), \quad g(x,y) = 1,
\]
donde \(g(x,y) = \norm{x}^2 + \norm{y}^2\).

Si hacemos \(x = x_1,\dots, x_m\) y \(y = y_1,\dots y_m,\)
el sistema de ecuaciones es el siguiente:
\[
\begin{pmatrix}
    y_1 \\ \vdots \\ y_m \\ x_1 \\ \vdots \\ x_m
\end{pmatrix}
= \lambda
\begin{pmatrix}
    2x_1 \\ \vdots \\ 2x_m \\ 2y_1 \\ \vdots \\ 2y_m
\end{pmatrix}
, \quad
\norm{x}^2+\norm{y}^2 = 1.
\]

De la primera ecuación se tiene que \(y_1 = \lambda 2x_1\),
al sustituir esto en la ecuación \(m+1\) se obtiene
\(x_1 = 2\lambda(2\lambda x_1) = 4\lambda^2x_1\).
Se puede continuar de esta forma, para obtener las
\(m\) ecuaciones siguientes:
\[
\begin{matrix}
    x_1 (1-4\lambda^2) = 0 \\
    \vdots \\
    x_m (1-4\lambda^2) = 0.
\end{matrix}
\]

De las ecuaciones anteriores se sigue que
\(x = 0\) o \(1-4\lambda^2 = 0\), es decir,
\(\lambda = \frac12\). Además, el procedimiento
anterior se puede hacer para obtener \(m\) ecuaciones
de la forma \(y_i (1-4\lambda^2) = 0\), por lo que
también hay que considerar el caso \(y = 0\).

Si \(\lambda=\frac12\) entonces se puede sustituir 
en el sistema de ecuaciones para obtener que \(x = y\).

Si \(x = 0\) o \(y = 0\) la restricción nos dice que
\(\norm{y} = 1\) o \(\norm{x} = 1\), respectivamente.

Los puntos críticos son de la forma \((x,x)\) o
\((0,y)\), con \(\norm{y} = 1\), o \((x,0)\) con
\(\norm{x} = 1\).

Ahora, como \(f(x,x) = \norm{x}^2\) y \(f(x,0) = f(y,0) = 0\)
se sigue que los puntos de la forma \((x,x)\) son valores
máximos de \(f\) y los puntos de la forma \((x,0)\) o
\((0,y)\) son valores mínimos. De esto se deduce la 
desigualdad de Cauchy-Schwarz, en la esfera unitaria:
\[
f(x,y) \leq f(x,x) \implies f(x,y)^2\leq f(x,x)^2,
\]
como la esfera es unitaria se tiene que \(f(x,x) = f(y,y) = 1\)
y
\[
f(x,y)^2 \leq f(x,x)f(y,y).
\]

\newpage
\begin{small}
    \begin{center}
    \begin{minipage}{0.4\paperwidth}
        \section*{Colophon \& Copyright}
        This document was typeset using \LuaTeX%
        \footnote{%
            \TeX\ is
            a typesetting software, free and open source,
            developed by Donald Knuth. \LaTeX\ is a macro
            set for \TeX\ developed by Leslie Lamport. \LuaTeX\ is
            a reworking of \TeX\ adding native support for the Lua
            programming language and unicode, among other things.
            All of them are available in all major
            operating systems.
        }
        and the \LaTeXe\ macros in a Linux operating system.
        The editor used for editing the text was Visual Studio Code%
        \footnote{%
            VS Code is an open source text editor
            developed by Microsoft.
            It's closer to an IDE than a traditional text editor
            like vi or emacs.
            The LaTeX Workshop extension provides the LaTeX
            integration for VS Code.
        }.
        The main typeface used are Spartan MB%
        \footnote{%
            Spartan is a geometric typeface
            by Matt Bailey. It's author describes it has
            ``An open-source typeface based on an ATF classic. 
            Built from necessity."
        }
        for text and TeX Gyre DejaVu Math for maths%
        \footnote{%
            In \LuaLaTeX\ this amounts to using the package
            \icode{unicode-math}
            and then setting the corresponding
            opentype fonts
        }.

        \medskip
        %
        \begin{quote}\ttfamily\raggedright
            E-mail: \url{jalb97@gmail.com} \\
            Copyright Jhonny Lanzuisi 2021\\
            This work is licensed under the Creative Commons Attribution-ShareAlike
            International (CC BY-SA 4.0)  License. To view a copy of the license,
            visit \url{https://creativecommons.org/licenses/by-sa/4.0/}.
        \end{quote}
        %
    \end{minipage}
    \end{center}
\end{small}
\newpage
\tableofcontents

\end{document}