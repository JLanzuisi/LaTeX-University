\documentclass{scrartcl}

% So that TeX doesn't complain about small
% underfull or overfull boxes
\hfuzz1pc
% Make the overfull marker bigger
\overfullrule=2cm

% Font setup.
\usepackage[neoeuler]{fontsetup}
\defaultfontfeatures{Ligatures=TeX}
\newfontfamily{\light}{League Spartan Light}
% 20% bigger line height
\linespread{1.04}
% Don't put extra space after periods
\frenchspacing

\KOMAoptions{
    paper = letter,
    BCOR = 0mm,
    twoside = false,
    fontsize = {10},
    DIV = calc,
}

% Make bibliography more compact, no indents.
\KOMAoption{toc}{flat}

% Language support, usually changes between english
% and spanish.
%\usepackage[spanish,es-noindentfirst]{babel}
\usepackage[english]{babel}
\usepackage{csquotes}

% Bibliography
\usepackage[
    backend=biber,
    style=numeric-comp,
    backref=true,
    backrefstyle=two,
    abbreviate=true
]{biblatex}
\addbibresource{~/git/Misc-LaTeX-files/bib/general.bib}
\addbibresource{~/git/Misc-LaTeX-files/bib/math-books.bib}

% Graphics, mainly to insert images or
% single page PDFs.
\usepackage{graphicx}
\usepackage[dvipsnames]{xcolor}
% Handy command to typeset URLs
\usepackage{hyperref}
\hypersetup{
    colorlinks=true,
    linkcolor=Mahogany,
    filecolor=Mahogany,
    urlcolor=Black,
    citecolor=Mahogany,
}
\usepackage{url}
%\urlstyle{same}
\usepackage{metalogo}


% Font style and size for title
\setkomafont{title}{\fontseries{m}\light}
% Font style for the subject
\setkomafont{subject}{\normalfont}
% Font style for subtitle
\setkomafont{subtitle}{\normalfont}
\setkomafont{author}{\large}
\setkomafont{date}{\normalsize}
\setkomafont{section}{\fontseries{m}\light\Large}
\setkomafont{subsection}{\fontseries{m}\light\large}
\setkomafont{subsubsection}{\fontseries{m}\normalfont}
% Footnotes
\deffootnote{2.0em}{1.5em}{\thefootnotemark.\ }

% CUSTOM MACROS
% ToC not in sans-serif
\let\oldtoc\tableofcontents
\renewcommand{\tableofcontents}{%
    {
        \renewcommand{\sffamily}{\rmfamily}
        \oldtoc%
    }
}
% Macros for writing code
\newcommand{\icode}[1]{\verb~#1~}
% Urls
% \let\oldurl\url
% \newcommand{\umark}{\text{\textcolor{Mahogany}{\texttt{o}}}}
% \renewcommand{\url}[1]{\oldurl{#1}\(^{\umark}\)}
% emph to bold
\renewcommand{\emph}{\textbf}
% math macros
\renewcommand{\Rn}{\mathbb{R}^{\mathrm{n}}}
\newcommand{\Rm}{\mathbb{R}^{\mathrm{m}}}
\newcommand{\R}{\mathbb{R}}
\newcommand{\N}{\mathbb{N}}

\begin{document}

%
\title{Segunda tarea}
\subtitle{Segunda evaluación del curso}
\subject{Análisis III}
\titlehead{Universidad Simón Bolívar\hfill Caracas, Venezuela}
\author{{\normalsize hecho por} \\ Jhonny Lanzuisi}
\date{\today}
\maketitle

\section{Primer ejercicio}

Sean \(A\subseteq\Rn\) abierto y conexo,
\(f\colon A\to\Rm\) diferenciable en \(A\)
tal que \(f'(x)=0\) para todo \(x\in A\).
Entonces, \(f\) es constante en \(A\).

\subsection{Solución}

Como \(f'(x) = 0\) se tiene,
de la definición de la derivada,
\[
\lim_{h\to0}
\frac{1}{\norm{h}}
\big(f(a+h)-f(a)\big) = 0.
\]

Esto implica que, dado \(\epsilon>0\)
existe \(\delta>0\) para el cual
\(\norm{h}<\delta\) implica
\[
\norm{\frac{1}{\norm{h}} f(a+h)-f(a)} < \epsilon
\]
pero esto es lo mismo que
\[
\norm{f(a+h)-f(a)} < \norm{h}\epsilon.
\]

La última ecuación implica que
el lado derecho de la ecuación se puede
hacer arbitrariamente pequeño,
es decir,
\[
f(a+h) - f(a) = 0,
\]
de donde se sigue que \(f\)
es constante, pues \(a\) es
un elemento cualquiera de \(A\).

\section{Segundo ejercicio}

Si \(f'(x_0)\) es inyectiva, existe un
entorno de \(x_0\) en el cual
\[x_0\neq x\implies f(x_0)\neq f(x)\]

\subsection{Solución}

Supongamos que para todo \(x_0\in A\),
y todo entorno de \(x_0\), \(f(x) = f(x_0)\).
Luego, como \(f\) es diferenciable en \(x_0\),
existe un \(x\) en el entorno de \(x_0\) tal que
\[
f(y) = f(x_0) - f'(x_0)(x-x_0)
\]
pero como \(f(y)=f(x_0)\),
\[
0 = f'(x_0)(x-x_0).
\]

La función \(f\) es por hipótesis inyectiva,
por lo que el \(\det(f'(x_0))\neq 0\).
A su vez esto implica que \(x = x_0\)
lo cual es una contradicción.

\section{Tercer ejercicio}

Sea \(f\colon A\to\Rn\) diferenciable en todo
\(A\subseteq\Rn\). Sean \(\varphi\colon \Rn\to A\times\Rn\)
y \(F\colon A\times\Rn\to\Rn\) donde
\[\varphi(x) = (x,f(x))\qquad F(x,y) = f(x)-y.\]

Muestre que \(F,\varphi\) son diferenciables,
diga quién es \(F', \varphi'\) y concluya que
el \(\ker(F)\) coincide con la imagen de \(\varphi'\)
y que \(\varphi'\) es inyectiva.

\subsection{Solución}

La función \(\varphi\) se escribe como
\(\varphi=(f_1,f_2)\) donde \(f_1\) es la función
identidad y \(f_2 = f\). Como ambas funciones son
diferenciables, se sigue que \(\varphi\) también lo es.

Por otra parte, \(F\) es la resta de dos funciones
\(f(x),y\)
diferenciables y, por lo tanto, es diferenciable. 

Como \(\varphi\) es diferenciable su derivada
es la matriz jacobiana. En este caso, como
\(\varphi\colon \Rn\to A\times\Rn\) se tiene que las dimensiones
de la matriz son \(2n\times n\).
Como la función identidad es una transformación lineal,
su matriz jacobiana es ella misma.
Tenemos entonces:
\[
\varphi' = 
    \begin{pmatrix}
    I_{n} \\
    f'
    \end{pmatrix}
\]
donde \(I_{n}\) es la identidad de tamaño \(n\)
y \(f'\) la matriz jacobiana de \(f\).
Nótese que, debido a que \(f\colon A\to\Rn\),
\(f'\) es de tamaño \(n\times n\) y \(\phi'\)
tiene entonces el tamaño correcto.

Ahora, como \(F\) es diferenciable su derivada
es su matriz jacobiana de tamaño
\(2\times 2n\). Supongamos que las funciones
componentes de \(f\) son \(f_1,\dots,f_n\)
y que \(y = (y_1,\dots,y_n)\).
Entonces, \(F\) se puede escribir de la siguiente
forma:
\[
F(x,y) = \big(f_1(x)-y_1, f_2(x)-y_2, \dots, f_n(x)-y_n\big).
\]

Al calcular las derivadas parciales, de la ecuación anterior,
se llega a la conclusión de que
\[
F' = 
    \begin{pmatrix}
    f' & -I_n
    \end{pmatrix}
\]
donde \(f'\) es el jacobiano de \(f\) y \(-I_n\)
el negativo de la matriz identidad de tamaño \(n\)
(este signo `-' proviene del \(-y\) en la definición de \(F\)).
Notemos que, nuevamente, la matriz \(F'\) tiene las dimensiones
correctas.

Como la matriz \(\varphi'\) tiene \(n\) pivotes entonces
es inyectiva. Más aún:
\[
\ker(F') = \img(\varphi') = 
    \gen \left\{
        (\varphi'_1), (\varphi'_2),\dots, (\varphi'_n)
    \right\}
\]
donde \(\varphi'_i\) son las columnas de la matriz \(\varphi'\).

\section{Cuarto ejercicio}

Sean \(A\subseteq\Rn\), \(f\colon A\times[a,b]\to\Rm\),
\(\partial_1 f\colon A\times[a,b]\to\mathcal{L}(\Rn,\Rm)\) continua
y \(\alpha,\beta\) de clase \(C^1([a,b])\).

Sea \(\Phi\colon A\to\Rm\),
\[\Phi(x) = \int_{\alpha(x)}^{\beta(x)} f(x,t) dt.\]

Muestre que \(\Phi\in C^1(A)\) 
y calcule \(\Phi'(x)h\), \(x\in A\) y \(h\in\Rn\).

\section{Quinto ejercicio}

Sea
\[\Phi(x,y) = \int_a^x f(x+y,t) dt.\]

Calcule
\[\devpart{\Phi}{x}, \devpart{\Phi}{y}.\]

\section{test}

\newpage
\begin{small}
    \begin{center}
    \begin{minipage}{0.4\paperwidth}
        \section*{Colophon \& Copyright}
        This document was typeset using \LuaTeX%
        \footnote{%
            \TeX\ is
            a typesetting software, free and open source,
            developed by Donald Knuth. \LaTeX\ is a macro
            set for \TeX\ developed by Leslie Lamport. \LuaTeX\ is
            a reworking of \TeX\ adding native support for the Lua
            programming language and unicode, among other things.
            All of them are available in all major
            operating systems.
        }
        and the \LaTeXe\ macros in a Linux operating system.
        The editor used for editing the text was Visual Studio Code%
        \footnote{%
            VS Code is an open source text editor
            developed by Microsoft.
            It's closer to an IDE than a traditional text editor
            like vi or emacs.
            The LaTeX Workshop extension provides the LaTeX
            integration for VS Code.
        }.
        The main typeface used are Spartan MB%
        \footnote{%
            Spartan is a geometric typeface
            by Matt Bailey. It's author describes it has
            ``An open-source typeface based on an ATF classic. 
            Built from necessity."
        }
        for text and TeX Gyre DejaVu Math for maths%
        \footnote{%
            In \LuaLaTeX\ this amounts to using the package
            \icode{unicode-math}
            and then setting the corresponding
            opentype fonts
        }.

        \medskip
        %
        \begin{quote}\ttfamily\raggedright
            E-mail: \url{jalb97@gmail.com} \\
            Copyright Jhonny Lanzuisi 2021\\
            This work is licensed under the Creative Commons Attribution-ShareAlike
            International (CC BY-SA 4.0)  License. To view a copy of the license,
            visit \url{https://creativecommons.org/licenses/by-sa/4.0/}.
        \end{quote}
        %
    \end{minipage}
    \end{center}
\end{small}
\newpage
\tableofcontents

\end{document}
