\documentclass{scrartcl}

% So that TeX doesn't complain about small
% underfull or overfull boxes
\hfuzz1pc
% Make the overfull marker bigger
\overfullrule=2cm

% Font setup.
\usepackage[neoeuler]{fontsetup}
\defaultfontfeatures{Ligatures=TeX}
\newfontfamily{\light}{League Spartan Light}
% 20% bigger line height
\linespread{1.04}
% Don't put extra space after periods
\frenchspacing

\KOMAoptions{
    paper = letter,
    BCOR = 0mm,
    twoside = false,
    fontsize = {10},
    DIV = calc,
}

% Make bibliography more compact, no indents.
\KOMAoption{toc}{flat}

% Language support, usually changes between english
% and spanish.
%\usepackage[spanish,es-noindentfirst]{babel}
\usepackage[english]{babel}
\usepackage{csquotes}

% Bibliography
\usepackage[
    backend=biber,
    style=numeric-comp,
    backref=true,
    backrefstyle=two,
    abbreviate=true
]{biblatex}
\addbibresource{~/git/Misc-LaTeX-files/bib/general.bib}
\addbibresource{~/git/Misc-LaTeX-files/bib/math-books.bib}

% Graphics, mainly to insert images or
% single page PDFs.
\usepackage{graphicx}
\usepackage[dvipsnames]{xcolor}
% Handy command to typeset URLs
\usepackage{hyperref}
\hypersetup{
    colorlinks=true,
    linkcolor=Mahogany,
    filecolor=Mahogany,
    urlcolor=Black,
    citecolor=Mahogany,
}
\usepackage{url}
%\urlstyle{same}
\usepackage{metalogo}


% Font style and size for title
\setkomafont{title}{\fontseries{m}\light}
% Font style for the subject
\setkomafont{subject}{\normalfont}
% Font style for subtitle
\setkomafont{subtitle}{\normalfont}
\setkomafont{author}{\large}
\setkomafont{date}{\normalsize}
\setkomafont{section}{\fontseries{m}\light\Large}
\setkomafont{subsection}{\fontseries{m}\light\large}
\setkomafont{subsubsection}{\fontseries{m}\normalfont}
% Footnotes
\deffootnote{2.0em}{1.5em}{\thefootnotemark.\ }

% CUSTOM MACROS
% ToC not in sans-serif
\let\oldtoc\tableofcontents
\renewcommand{\tableofcontents}{%
    {
        \renewcommand{\sffamily}{\rmfamily}
        \oldtoc%
    }
}
% Macros for writing code
\newcommand{\icode}[1]{\verb~#1~}
% Urls
% \let\oldurl\url
% \newcommand{\umark}{\text{\textcolor{Mahogany}{\texttt{o}}}}
% \renewcommand{\url}[1]{\oldurl{#1}\(^{\umark}\)}
% emph to bold
\renewcommand{\emph}{\textbf}
% math macros
\renewcommand{\Rn}{\mathbb{R}^{\mathrm{n}}}
\newcommand{\Rm}{\mathbb{R}^{\mathrm{m}}}
\newcommand{\R}{\mathbb{R}}
\newcommand{\N}{\mathbb{N}}

\begin{document}

%
\title{Primera tarea}
\subtitle{Primera evaluación del curso}
\subject{Análisis III}
\titlehead{Universidad Simón Bolívar\hfill Caracas, Venezuela}
\author{by \\ Jhonny Lanzuisi}
\date{\today}
\maketitle

\section{Primer ejercicio}

Demuestre que los intervalos \([a,b]\) son conjuntos conexos
en \(\R\).

\subsection{Solución}

Supongamos que el intervalo cerrado no es conexo.
Entonces existen dos conjuntos abiertos \(U,V\) que forman
una separación de \([a,b]\), es decir,
\(U\cap[a,b]\neq\emptyset\), \(V\cap[a,b]\neq\emptyset\),
\([a,b]\cap U\cap V = \emptyset\) y \([a,b]\subset U\cup V\).

Supongamos que \(b\in V\).
El intervalo \(U\cap[a,b]\) esta acotado superiormente
por lo que tiene un supremo, llamémosle \(c\).
Además, este último conjunto es cerrado pues
su complemento, dado por \(V\cup(\Rn\setminus[a,b])\),
es abierto al ser la unión de dos conjuntos abiertos.
Esto implica que \(c \in U\cap[a,b]\).

Tenemos además que \(c\neq b\) pues \(c\not\in V\) y
\(b\in V\).

Ahora, como \(c\neq b\) y ningún entorno de \(c\)
puede estar totalmente contenido en \(U\), tenemos
que cualquier entorno de \(c\) intersecta al conjunto \(V\cap[a,b]\).
Pero esto significa que \(c\) es un punto de acumulación
de \(V\cap[a,b]\). Al ser \(V\cap[a,b]\) cerrado, por una razón
similar a la dicha antes para \(U\cap[a,b]\), contiene a todos
sus puntos de acumulación y \(c \in V\cap[a,b]\).

Pero entonces \(c\) es un elemento de \(V\cap[a,b]\) y \(U\cap[a,b]\),
lo que contradice el hecho de que \([a,b]\cap U\cap V = \emptyset\).
\section{Segundo ejercicio}

Demuestre que todo conjunto \(A\) conexo por caminos es conexo.

\subsection{Solución}

Supongamos que \(A\) no es conexo.
Entonces, como en el ejercicio anterior,
existen dos conjuntos abiertos \(U,V\)
que separan a \(A\).

Tomemos un \(x \in U\) y \(y\in V\).
Como \(A\) es conexo por caminos, existe un camino
\(\phi \colon [a,b]\to \Rn\) contenido en \(A\) tal que
\(\phi(a) = x\) y \(\phi(b) = y\).

Hagamos \(U_0=\phi^{-1}(U)\) y \(V_0 = \phi^{-1}(V)\),
entonces tenemos que tanto \(U_0\) como \(V_0\) son
subconjuntos de \([a,b]\).

Primero, los conjuntos \(U_0, V_0\) son abiertos
puesto que \(\phi\) es continua, y las imágenes
inversas de conjuntos abiertos por funciones continuas
son abiertas.

Segundo, es claro que \(U_0\cap[a,b]\neq\emptyset\)
y \(V_0\cap[a,b]\neq\emptyset\) puesto que
\(\phi^{-1}(x) = a \in U_0\) y, de la misma forma,
\(\phi^{-1}(y) = b \in V_0\).

Tercero, \([a,b] \subseteq U_0\cup V_0\).
Para ver esto supongamos lo contrario,
que existe un \(c\in [a,b]\) que no pertenece
ni a \(V_0\) ni a \(U_0\).
Entonces \(\phi(c)\) no pertenece ni a \(U\) ni
a \(V\) pero si pertenece a \(A\), pues \(A\)
es conexo por caminos. Y esto implica que
\(A\not\subset U\cup V\), lo cual es una contradicción,
pues \(U,V\) forman una separación de \(A\).

Cuarto, \([a,b]\cap U_0\cap V_0 = \emptyset\).
Para demostrar este hecho, supongamos lo contrario:
existe un \(c\in[a,b]\cap U_0\cap V_0\). Pero entonces,
\(\phi(c)\in V\), \(\phi(c)\in U\) y \(\phi(c)\in A\)
contradiciendo el hecho de que \(A\cap U\cap V = \emptyset\).

Los cuatro hechos demostrados anteriormente
implican que \(U_0,V_0\) forman una separación de
\([a,b]\). Pero esto es imposible, por el ejercicio anterior.
Entonces \(A\) tiene que ser conexo.

\section{Tercer ejercicio}

Sean \(f\colon A\subseteq\Rn\to\Rm\) y \(x_0\in A^*\).
Entonces
\[\lim_{x\to x_0}f(x) = L\]
si, y solo si, para cada sucesión \(\{x_k\}\subset A\)
que converge a \(x_0\)
\[\lim_{k\to\infty} f(x_k) = L.\]

\subsection{Solución}

\subsubsection{Primera implicación}

Supongamos que el límite \(\lim_{x\to x_0}f(x) = L\),
que \(x_k\to x_0\) y sea \(\epsilon>0\).
Queremos hallar un natural \(N\) para el cual \(k\geq N\)
implica \(d(f(x_k),L) < \epsilon\).
Como el límite de \(f\) existe, existe un \(\delta >0\) tal que
si \(d(x,x_0) < \delta\)
entonces \(d(f(x),L) < \epsilon\).
Como la sucesión $x_k$ converge a $x_0$, existe $N$ tal que
\(d(x_k,x_0) < \delta\).
Entonces, para todo \(k\geq N\) se tendrá que
\(d(f(x_k),L) < \epsilon\).

\subsubsection{Segunda implicación}

Supongamos ahora que para toda sucesión \(x_k\to x_0\)
de \(A\) se tiene que \(f(x_k)\to f(x_0)\).
Supongamos que $L$ no es el límite de la función \(f\)
cuando $x\to x_0$.
Esto significa que existe un \(\epsilon > 0\) tal que
para cada \(k\in\N\) existe un \(x_k\in A\) tal que
\[ 0 < d(x_k,x_0) < \frac{1}{k} \]
y
\[ d(f(x_k),L) \geq \epsilon.\]

Entonces la sucesión \(x_k\) converge a \(x_0\)
pero la sucesión \(f(x_k)\) no converge a \(L\),
lo cual es una contradicción.

\section{Cuarto ejercicio}

Sean \(A\subseteq\Rn\), \(x_{0}\in A^{\ast}\)
y \(f\colon A \to\Rn\) tal que
\(\lim_{x\to x_{0}} f(x)\)  existe.
Luego, existen \(\epsilon>0\) y \(M>0\) tales
que \(\left\lVert f(x) \right\rVert_{\Rn} < M\)
para todo \(x\in B(x_0,\epsilon)\).

\subsection{Solución}

Por la definición de límite,
dado \(\epsilon=1\) existe \(\delta>0\)
tal que, si \(\lVert x-x_0\rVert<\delta\)
entonces
\[\lVert f(x) - L \rVert < 1.\]
Por la desigualdad triangular,
\[\lVert f(x)\rVert \leq \lVert f(x) - L \rVert + \lVert L\rVert 
    < 1 + \lVert L\rVert\]
por lo que basta con tomar \(M=1+\lVert L\rVert\).

\section{Quinto Ejercicio}

Sea \(A\subseteq\Rn\) un compacto y \(f\colon A\to\Rm\)
una función continua sobre \(A\). Entonces, \(f(A)\) es
uniformemente continua.

\subsection{Solución}

Sea \(\epsilon>0\). Queremos encontrar \(\delta>0\)
tal que 
\[d(x,y)<\delta\implies d(f(x),f(y)) < \epsilon.\]

Como \(f\) es continua en cada \(x\in A\) existe un
\(\delta_k>0\) tal que 
\[f(B(x,\delta_k))\subseteq B(f(x),\frac{\epsilon}{2}).\]
Pero \(\{B(x_i,\frac{\delta_x}{2})\}_{x\in A}\) es un cubrimiento
abierto de \(A\), y como \(A\) es compacto existe un subcubrimiento
finito \(\{ B(x_i,\frac{\delta_x}{2})\}_{i=1}^n\).

Ahora, sea \(\delta = \min(\frac{\delta_{x_i}}{2})\).
Entonces, si \(d(x,y) < \delta\)
\[d(f(x),f(y))\leq d(f(x),f(x_i)) + d(f(x_i),f(y)) < \frac{\epsilon}{2} + \frac{\epsilon}{2} = \epsilon.\]

\section{Sexto Ejercicio}

Sea \(E\subseteq\Rn\) y \(f \colon E\to\Rm\) una función continua.
Si \(E\) es conexo entonces \(f(E)\) también es conexo.

\subsection{Solución}

Supongamos que \(f(E)\) no es conexo. Entonces existen
abiertos \(U,V\) que separan a \(f(E)\), es decir,
\(f(E)\subset V\cup U\), \(f(E)\cap V\neq\emptyset\),
\(f(E)\cap U\neq\emptyset\) y \(f(E)\cap U\cap V = \emptyset\).

Como los conjuntos \(U,V\) son abiertos y la función
\(f\) es continua, sus preimágenes por \(f\) deben ser
también conjuntos abiertos, es decir, \(U'=f^{-1}(U)\) y \(V'=f^{-1}(V)\)
son ambos conjuntos abiertos.
Además, podemos aplicar \(f^{-1}\) a las expresiones
del párrafo anterior que definen la separación de \(f(E)\)
para obtener que \(U',V'\) forman una separación de \(E\). Por ejemplo,
\(f^{-1}(f(E))\subset f^{-1}(U)\cup f^{-1}(V)\).
Pero esto es una contradicción puesto que \(E\) es conexo.

%\printbibliography
\newpage
\begin{small}
    \begin{center}
    \begin{minipage}{0.4\paperwidth}
        \section*{Colophon \& Copyright}
        This document was typeset using \LuaTeX%
        \footnote{%
            \TeX\ is
            a typesetting software, free and open source,
            developed by Donald Knuth. \LaTeX\ is a macro
            set for \TeX\ developed by Leslie Lamport. \LuaTeX\ is
            a reworking of \TeX\ adding native support for the Lua
            programming language and unicode, among other things.
            All of them are available in all major
            operating systems.
        }
        and the \LaTeXe\ macros in a Linux operating system.
        The editor used for editing the text was Visual Studio Code%
        \footnote{%
            VS Code is an open source text editor
            developed by Microsoft.
            It's closer to an IDE than a traditional text editor
            like vi or emacs.
            The LaTeX Workshop extension provides the LaTeX
            integration for VS Code.
        }.
        The main typeface used are Spartan MB%
        \footnote{%
            Spartan is a geometric typeface
            by Matt Bailey. It's author describes it has
            ``An open-source typeface based on an ATF classic. 
            Built from necessity."
        }
        for text and TeX Gyre DejaVu Math for maths%
        \footnote{%
            In \LuaLaTeX\ this amounts to using the package
            \icode{unicode-math}
            and then setting the corresponding
            opentype fonts
        }.

        \medskip
        %
        \begin{quote}\ttfamily\raggedright
            E-mail: \url{jalb97@gmail.com} \\
            Copyright Jhonny Lanzuisi 2021\\
            This work is licensed under the Creative Commons Attribution-ShareAlike
            International (CC BY-SA 4.0)  License. To view a copy of the license,
            visit \url{https://creativecommons.org/licenses/by-sa/4.0/}.
        \end{quote}
        %
    \end{minipage}
    \end{center}
\end{small}
\newpage
\tableofcontents

\end{document}