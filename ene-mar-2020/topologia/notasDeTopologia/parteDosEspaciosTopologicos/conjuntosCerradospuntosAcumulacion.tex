\chapter{Conjuntos cerrados y puntos de acumulación}%
\label{cha:Conjuntos cerrados y puntos de acmulación}

\section{Espacios de Hausdorff}%
\label{sec:Espacios de Hausdorff}

El conjunto de los números reales (y $\mathbb{R}^2$) con la topología usual
tienen propiedades que no cumple cualquier espacio topológico. Por ejemplo,
en $\mathbb{R}$ todo cojunto de un solo elemento es cerrado; esto sin embargo no 
es cierto para otros espacios topológicos.

Otro ejemplo lo dan las sucesiones convergentes. Una sucesión en un espacio topológico es
una función de los naturales a este espacio. Decimos que una sucesión \Keyword{converge}
a un punto $x$ del espacio si, para cada entorno de $x$, existe un número entero $N$
tal que la cola (los $n>N$) de la sucesión esta totálmente contenida en dicho entorno.
En $\mathbb{R}$ las sucesiones no puede converger a dos puntos distintos, sin embargo,
en espacios mas generales esto si puede ocurrir.

Los espacios que se comportan como $\mathbb{R}$ en el sentido antes mencionado reciben un
nombre especial:

\begin{defi}
	Un espacio topológico $X$ es llamado de \Keyword{Hausdorff} si para cada par
	de puntos distintos $x_{1},x_2$ de $X$ se pueden encontrar dos entornos de
	dichos puntos que son disjuntos.
\end{defi}

\begin{teo}
	Todo cojunto de una cantidad finita de puntos en un espacio de hausdorff es cerrado.
\end{teo}

\begin{proof}
	Basta hacer la demostración para los conjuntos que tienen un solo elemento.
	Sea $x$ un punto cualquiera de un espacio de hausdorff $X$. Entonces, si tomamos cualquier otro punto
	$x'\in X$ la condición de hausdorff nos dice que existe un entorno de $x'$ que no intersecta a $\{x\}$. Luego,
	$x$ es el único punto tal que cada entorno intersecta al conjunto $\left\{ x \right\}$ y, por lo tanto,
	$x$ es su propia clausura y es cerrado.
\end{proof}
