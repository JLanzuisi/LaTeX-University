\chapter{Propiedades}

En nuestra definición de conjunto aludimos a unas
\emph{propiedades} que los elementos del conjunto
compartían. Tenemos ahora la tarea de establecer
ciertas reglas con las que podamos enunciar estas
propiedades, con el fin de evitar ambigüedades. 

Las reglas que vamos a explicar son, en esencia, las de
la \emph{lógica}. Si se quiere un estudio riguroso de
estas reglas será mejor remitirse a un libro de lógica
matemática, aquí se hablará de los conceptos de manera
informal.

La relación más básica en la teoría de conjuntos es la
de \emph{pertenencia}, que denotamos con el símbolo
$\in$. La expresión $X\in Y$ se lee `$X$ pertenece a
$Y$' o `$X$ es un miembro de $Y$'.

Las letras $X$ e $Y$ usadas en el párrafo anterior son
\emph{variables}, denotan cualquier par de conjuntos.
La proposición `$X\in Y$' es verdadera o falsa
dependiendo de cuales son los conjuntos $X$ e $Y$.

Todas las demás propiedades de la teoría de grupos se
pueden expresar usando la pertenencia y algunas
herramientas lógicas: identidad, conectividad y
cuantificadores.

Hay veces en las que conviene expresar el mismo
conjunto con variables distintas, la relación de
igualdad ---o identidad--- de conjuntos la denotaremos
con el símbolo `$=$'.

\begin{ejem} Este ejemplo da varios hechos sobre la
	igualdad de conjuntos. Sean $X,Y$ y $Z$ tres
	conjuntos, entonces se cumple que:
	\begin{enumerate} 
		\item $X=X$.
		\item Si $X=Y$ entonces $Y=X$.
		\item Si $X=Y$ y
$Y=Z$, entonces $X=Z$.  
		\item Si $X=Y$ y $X\in Z$
entonces $Y\in Z$.
		\item Si $X=Y$ y $Z\in X$ entonces
$Z\in Y$.  \end{enumerate} \end{ejem}


