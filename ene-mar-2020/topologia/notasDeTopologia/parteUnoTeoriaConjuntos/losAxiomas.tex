\chapter{Los axiomas}%
\label{sec:los_axiomas}

En esta sección asentaremos las bases axiomáticas de la
teoría que queremos desarrollar. Se intentará dar la
intuición detrás de cada axioma, con el fin de que su
significado sea lo más claro posible.

El primer axioma que vamos a enunciar dice que existe
al menos un conjunto, y por lo tanto toda la discusión
hasta este punto no ha sido en vano.

\paragraph{Axioma de existencia}%
\label{par:axioma_de_existencia}

Existe un conjunto que no posee elementos.

Existen muchas maneras de denotar intuitivamente el
conjunto vacío, por ejemplo, pensemos en el conjunto de
todos los números enteros que son pares e impares al
mismo tiempo, o el conjunto de todos los reales $x$
tales que $x^{2}=-1$. Existen muchos ejemplos de esta
forma. Nuestra intuición no dice que este conjunto que
no tiene elementos ha de ser uno solo, pero aún no
podemos demostrar este hecho, nos hará falta el axioma
siguiente para poder caracterizar a un conjunto por sus
elementos.

\paragraph{Axioma de extensión}%
\label{par:axioma_de_extensión}

Sean $X,Y$ dos conjuntos, si todo elemento de $X$ es un
elemento de $Y$ y todo elemento de $Y$ es un elemento de
$X$ entonces $X=Y$.

