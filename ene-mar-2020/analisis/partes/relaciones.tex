\chapter{Relaciones}%
\label{cha:Relaciones}

\begin{primerpar}
	Antes de definir una función es útil
comentar la idea de \emph{relación} en profundidad, pues esta será
la base en la que se construirán las funciones. También se discutirán
las relaciones de orden, cuya importancia se verá en la parte~\ref{prt:topologia}-Topología.
\end{primerpar}

\section{Relaciones y sus propiedades}%
\label{sec:RelacionesPropiedades}


Una \Keyword{relación} es, de forma\textsf{a} intuitiva, una regla que nos permite
asignarle a elementos de un conjunto elementos de \emph{otro conjunto} (los conjuntos pueden ser el mismo, en cuyo caso es una relación de un conjunto en si mismo).
Decimos entonces que aquellos elementos que asigna esta regla están relacionados.

\begin{nota}
	Esto e suna prueba de una nota, algo de mate $x=z+r^2$ y mas palabras pa probar
\end{nota}

Consideremos unos cuantos ejemplos antes de embarcarnos en la definición rigurosa de
una relación.

\begin{ejem}
	La relación padre-hijo es una relación en el sentido que las estamos estudiando: si consideramos
	los conjuntos de las personas mayores de $30$ años y las personas menores de $30$ entonces
	podemos relacionar a dos individuos, si uno es padre del otro, o si uno es hijo del otro.
\end{ejem}

\begin{ejem}
	Tomemos los conjuntos finitos $\mathcal{A}=\{ a,b,c \}$ y $\mathcal{B}=\{ d,e,f \}$.
	Una relación es pues cualquier regla de asignación entre estos dos conjuntos. Por ejemplo, la regla
	que asigna a cada elemento de $\mathcal{A}$ el elemento $e$ de $\mathcal{B}$ es una relación.
\end{ejem}

\marginnote{Notación Importante}
Si llamamos a nuestra relación $R$ entonces el ejemplo anterior puede escribirse como: $xRe$ para todo $x\in\mathcal{A}$.

\begin{ejem}\label{ejem:nofun}
	Consideremos los conjuntos finitos $\mathcal{A}=\{ a \}$ y $\mathcal{B}=\{ b,c \}$. La relación $R$ dada por
	$aRb$ y $aRc$ es una relación entre los dos conjuntos. Nótese que en este caso al elemento $a$ se le asignan
	dos elementos \emph{distintos} de $\mathcal{B}$.
\end{ejem}

\begin{ejem}\label{ejem:sifun}
	Tomemos el conjunto de los números reales. Entonces una relación de $\mathbb{R}$ en $\mathbb{R}$
	viene dada por $xR2x$ para todo $x\in\mathbb{R}$.
\end{ejem}

Tomando en cuenta los ejemplos anteriores podemos escribir la relación $aRb$ mediante la notación $(a,b)$.
Se vuelve entonces evidente que estamos tratando con elementos de un \emph{producto cartesiano}: si
$a\in\mathcal{A}$ y $b\in\mathcal{B}$ entonces estamos tomando elementos de $\mathcal{A}\times\mathcal{B}$.
Esta es la idea central de la definición.

\begin{defi}
	Sean $\mathcal{A},\mathcal{B}$ dos conjuntos (no necesariamente distintos).
	Una \Keyword{relación} entre ellos es cualquier subconjunto del producto cartesiano
	$\mathcal{A}\times\mathcal{B}$.
\end{defi}

Otra forma de decirlo es que una relación entre los dos conjuntos es cualquier elemento de $\mathcal{P}(\mathcal{A}\times\mathcal{B})$.

Existen diversas propiedades que pueden cumplir las relaciones que nos son de interés, en lo que sigue `$R$' siempre denota una relación:
\begin{description}
	\item[Reflexividad] Si para todo $x$ se cumple que $xRx$.
	\item[Simetría] Si, dados dos elementos $x,y$ se cumple que
		$xRy$ implica $yRx$.
	\item[Transitividad] Si dado tres elementos $x,y,z$ se cumple que
		$xRy$ y $yRz$ implica $xRz$.
\end{description}

\begin{ejem}
	La relación en el ejemplo~\ref{ejem:sifun}, página~\pageref{ejem:sifun}, no es transitiva ni reflexiva ni simétrica.	
\end{ejem}

El nombre de alguna de estas propiedades le será familiar al lector de cuando discutimos la relación de orden de los números reales.
No será entonces una sorpresa que la noción de relación pueda usarse para definir \emph{ordenes.} Estos podrán ser \emph{parciales} o
\emph{totales}, como veremos a continuación.

\section{Relaciones de orden}%
\label{sec:Relaciones de orden}

