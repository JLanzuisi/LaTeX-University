\chapter{Notación}%
\label{cha:Notación}

%\begin{primerpar}
	Toda la notación importante del libro se explica aquí, para evitar confusiones.
	Todos los capítulos tienen al inicio un párrafo como este que explica y resume el capítulo.
	El tipo de fuente es mas grande para que sea fácil de ubicar.
%\end{primerpar}
\P~
%\begin{nota}
	Los comentarios en el texto estarán en cursiva y alineados a la izquierda, a veces serán aclaraciones y otras referencias históricas. El lector puede saltarse los comentarios si así lo desea, pues no son esenciales.
Para obtener información sobre las fuentes usadas, y el diseño de las notas en general, véase el apéndice~\ref{cha:Colofón}.
%\end{nota}
\P~
%\marginnote{Párrafo importante aquí}
Las señalizaciones en los márgenes ayudan a ubicar lugares importantes en la página de forma rápida. Un ejemplo de puede ver en la nota a este párrafo
\P~
Algunas porciones del texto son enlaces, por ejemplo la referencia a los apéndices en la nota anterior. Los las referencias
a otras partes del texto serán por lo general enlaces, así como los \textsc{url} y las referencias bibliográficas. Usarlos
ayudará bastante al lector.

\section*{Lista de símbolos}%
\label{sec:Lista de símbolos}

\begin{description}
	\item[$\mathbb{N},\mathbb{Z},\mathbb{Q},\mathbb{R},\mathbb{C}$] Los conjuntos de los números naturales,
		enteros, racionales, reales y complejos.
	\item[$\mathcal{A},\mathcal{B},\mathcal{C},\dots$] Conjuntos.
	\item[$\log x$] Logarítmo natural de $x$, es decir, tomando como base el número de euler $\mathrm{e}$.
\end{description}
