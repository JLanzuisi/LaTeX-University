\chapter{Números reales, Racionales y Enteros}%
\label{cha:números_reales_racionales_y_enteros}
\reversemarginpar
%\Clase{09/01}
%\vspace*{-.5\baselineskip}
%\begin{primerpar}
	\textsc{Cualquier intento de comprender} el análisis matemático 1983 
debe comenzar por tener una noción relativamente rigurosa de lo que es un número.
Esta capítulo definirá de manera axiomática (es decir, dando una lista de propiedades)
los números reales.
\P~
Dicho esto, no nos detendremos en fundamentar las nociones de \emph{número entero} 
y \emph{número racional}, más bien asumiremos que se esta familiarizado con ambos
sistemas numéricos.
\P~
	Entendemos por número entero, como es de esperar, los \emph{números de contar}
	y sus negativos, junto con el cero, denotados por
	\[
		\dots,-4,-3,-2,-1,0,\,1,\,2,\,3,\,4,\dots.
	\]
	Por \emph{número racional} entenderemos los números de la forma 
	\[
		\frac{p}{q} 
	\]
	donde $p$ y $q$ son ambos enteros y $q \neq 0$ .
\P~
El sistema de números racionales es inadecuado para muchos propósitos pues contiene,
en cierto sentido, \emph{huecos}. Por ejemplo, y una demostración de este hecho se puede
encontrar en los ejercicios,
no existe ningún número racional $q$ tal que $q^2=2$. Esto lleva a la inclusión de los
números \emph{irracionales} y la construcción de los números reales.
\P~
Dicho esto, no todo es malo con los números racionales. Este sistema es a la vez un 
\emph{cuerpo} y un \emph{conjunto ordenado} --- la definición de estos términos se dará
más adelante en este capítulo.
\P~
A partir de ahora nos referiremos a los números enteros y los racionales con los
símbolos $ \mathbb{Z}, \mathbb{Q}$ respectivamente.

\section{Definición axiomática de los reales}%
\label{sec:definición_axiomática_de_}

Existen muchas formas de definir los números reales, la más
común es construirlos a partir de los enteros y los racionales
(mediante \emph{cortes de dedekind} por ejemplo).
Sin embargo la definición dada aquí no es esa, la definición que daremos
es una lista de las propiedades fundamentales de los reales,
como se verá a continuación.
\P~
Las primeras 5 propiedades son las que corresponden a la definición
de un \emph{cuerpo}. La adición se denotará con el símbolo $+$ y la
multiplicación simplemente concatenando los valores a multiplicar.

\begin{description}
	\item[Conmutatividad]%

Sean $x,y\in \mathbb{R}$ entonces
\[
	x+y=y+x\quad\text{y}\quad xy=yx.
\]

\item[Asociatividad]

Sean $x,y,z\in \mathbb{R}$ entonces
\[
	x+(y+z)=(x+y)+z\quad\text{y}\quad x(yz)=(xy)z
\]

\item[Distributividad]%

Sean $x,y,z\in \mathbb{R}$ entonces
\[
	x(y+z) = xy+xz
\]

\item[Inversos y neutros aditivos]%

Dados dos números reales $x,y$  existe un número real $z$ tal que
\[
	x+z=y
\]
a este $z$ se le denota por $y-x$. El número $x-x=0$ es llamado el
\emph{cero} (la aparente dependencia de $x$ es irrelevante) y tiene la
propiedad de que para todo $a\in \mathbb{R}$, $a+0=a$. En vez de
$0-x$ escribiremos simplemente $-x$ y lo llamaremos el \emph{negativo} de $x$.

\item[Inversos y neutros multiplicativos]

Existe al menos un número real $x\neq0$. Dados dos números reales
$x,y$  existe un número real $z$ tal que
\[
	xz=y
\]
este $z$ esta denotado por $y/x$. El número $x/x=1$ es el 
\emph{neutro de la multiplicación} (cuya aparente
dependencia de $x$ es irrelevante). Escribiremos $x ^{-1}$ en 
vez de $1/x$ y lo llamaremos el \emph{recíproco} de $x$.


\end{description}

De estas cinco propiedades pueden derivarse todas las leyes usuales
de la aritmética.
\P~
Las cuatro propiedades siguientes tienen que ver con la
\emph{noción de orden} que existe entre los números naturales.
Asumimos que existe una relación $<$ con las cuatro propiedades
siguientes.

\begin{description}
	\item[Tricotomia]%

Una, y solo una, de las siguientes relaciones se satisface para
cualquier par de números reales $x,y$ 
\[
	x=y,\quad x<y,\quad x>y.
\]

\item[Propiedad 2]%

Si $x,y\in\mathbb{R}$ son tales que $x<y$ entonces, para cualquier
$z$ real,
\[
	x+z<y+z.
\]

\item[Propiedad 3]%

Si $x>0$ y $y>0$ entonces 
\[
	xy>0
\]

\item[Transitividad]%

Sean $x,y,z\in \mathbb{R}$. Si $x>y$ y $y>z$ entonces $x>z$.

\end{description}


\section{Ejercicios}%
\label{sec:ejercicios_reales}

\begin{ejer}
	Demostrar que $ \sqrt{2} $ es irracional
	\begin{sol}
		Haremos un argumento por contradicción. Supongamos
		que existen enteros $ p,q $ tales que
		\[
			\frac{p}{q} = \sqrt{2}
		\]
		y, además, que $ p,q $ no tienen divisores comunes
		(esta suposición no pierde generalidad, pues de tener
		divisores comunes estos simplemente se cancelarían al 
		dividir).
\P~
		Entonces nuestra suposición inicial implica
		\[
		\frac{p^2}{q^2} = 2.
		\]
		De donde se obtiene
		\[
			p^{2} = 2 q^{2} 
		\]
		y entonces $ p^{2} $ es par. Pero si $ p^{2}  $ es par 
		entonces $ p $ también es par. Como $ p $ es par se sigue
		que $ p^{2}  $ es de la forma $ 4k $ para algún $k\in \mathbb{Z}$
		y entonces, por la ecuación anterior, $q^{2} $ también es par
		y $ q $ es par.
\P~
		Pero esto contradice la suposición de que $ p,q $ no tenían divisores
		comunes. Por lo tanto no existen racionales $ p,q $ que satisfacen
		la primera ecuación.
	\end{sol}
\end{ejer}

\begin{ejer}
	Demostrar que $ \sqrt{2}+\sqrt{3} $ es irracional. 
	\begin{sol}
	Supongamos, buscando una contradicción, que $ \sqrt{2}+ \sqrt{3} $ es racional.
	Entonces su cuadrado debe ser un número racional, pues el producto de 
	dos racionales es racional, pero
	\[
		 ( \sqrt{2}+\sqrt{3} )^2 = 5+ \sqrt{6}
	\]
	es irracional, pues $ \sqrt{6} $ es irracional (la demostración
	de que $ \sqrt{6} $ es irracional es completamente análoga a la de que
	$ \sqrt{2} $ lo es, veáse el ejercicio anterior).
	\end{sol}
\end{ejer}

\begin{ejer}
	Demostrar las siguientes afirmaciones sobre los números reales
	($ a,b,c $ y $ d $ representan números reales).
	\begin{enumerate}
		\item Si $ a<b $ y $ c<d $ entonces $ a+c<b+d $.
		\item Si $ a<b $ entonces $ -b<-a $.
		\item Si $ a<b $ y $ c>d $ entonces $ a-c<b-d $.
		\item Si $ a<b $ y $ c>0 $ entonces $ ac<bc $
	\end{enumerate}
	\begin{sol}
		Veamos cada parte.
		\begin{enumerate}
			\item Nuestra hipótesis implica que
				\[
					a+c<b+c
				\]
				pero 
				\[
					b+c<b+d
				\]
				y por transitividad
				\[
					a+c<b+d.
				\]
			\item Sumando cero a ambos lados convenientemente
				se obtiene
				\[
					a-b+b<b-a+a
				\]
				y al restar $ a+b $ a ambos lados
				\[
					-b<-a.
				\]
			\item Por la parte anterior tenemos que $ c>d $ implica
				$ -c<-d $, y por la parte \textsc{(i)} se obtiene 
				\[
					a-c<b-d.
				\]
			\item Como $ a<b $ se tiene que $ a=b-u $ con $ u=b-a $,
				entonces
				\[
					ac = (b-u)c = bc-uc
				\]
				y esto implica que $ ac<bc $.
		\end{enumerate}
	\end{sol}
\end{ejer}

