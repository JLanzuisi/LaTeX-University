\documentclass[mid,fleqn,final,oneside]{tareas}
\clase{topología 1}
\DeclareMathOperator{\id}{id}
\begin{document}
\chapter[Ejercicios: Espacios topológicos y continuidad]{Ejercicios: Espacios topológicos y continuidad}%
\label{cha:ejercicios_sobre_espacios_topológicos}

\reversemarginpar\identi{Jhonny Lanzuisi}{15\,10759}
\vspace{-.3\baselineskip}
\section*{Página 81, ejercicio 1}%

Sea $ \{ U_i \} $ la familia de todos los conjuntos
abiertos $ U $ de $X$ tales que para cada $x\in A$,
$x\in U$ y $U\subset A$.  Entonces, como $ U_i\subset A
$ para todo $ i $, el conjunto $ A $ puede escribirse
como 
\[ A=\bigcup U_i \] 
y dado que los $ U_i $ son
conjuntos abiertos su unión ha de ser un conjunto
abierto.  Luego $ A $ es un conjunto abierto como se
buscaba.

\section*{Página 81, ejercicio 3}%

Primero que todo, el conjunto vacío pertenece a $
\mathcal{T}_c $ debido a que $ X\setminus\varnothing=X
$ y $X\in \mathcal{T}_c$ por definición.  También se
tiene que $ X $ es un conjunto abierto puesto que $
X\setminus X=\varnothing $ y el conjunto vacío es
numerable por ser finito.

\reversemarginpar\marginnote{Este tipo de igualdades se
siguen de las leyes de De Morgan} 
Supongamos que $
\left\{ U_k \right\}$ es una familia de elementos de $
\mathcal{T}_c $.  Entonces 
\[
	X\setminus\bigcup U_k= \bigcap (X\setminus U_k).
\]
\normalmarginpar\marginnote{Véase el Corolario 7.3, pag 46.}%
Pero el lado derecho de la igualdad es numerable puesto
que estas intersecciones son subconjuntos de todos los
$ X\setminus U_k $ y estos últimos son numerables.

Supongamos ahora que $ \left\{ U_1,\dots,U_n \right\} $
son una cantidad finita de elementos de $
\mathcal{T}_c. $ Entonces
\[
	X\setminus\bigcap_{i=1}^{n}U_i =
	\bigcup_{i=1}^{n} X\setminus U_i.
\]
\normalmarginpar\marginnote{Véase el teorema 7.5, pag 46}
Donde el lado derecho de la igualdad es numerable pues
la unión de conjuntos numerables es numerable.

Hemos visto que las uniones arbitrarias y las
intersecciones finitas de elementos de $\mathcal{T}_c$
pertenecen nuevamente a $\mathcal{T}_c$, esto es, que
$\mathcal{T}_c$ es una topología sobre $X$.

En el caso del conjunto $ \mathcal{T}_\infty $ se tiene
que \emph{no es} una topología sobre $ X $ en general.
Consideremos el siguiente contraejemplo: Hagamos $
X=\mathbb{Z} $ y consideremos los subconjuntos $
U_1,U_2 $ de
$ \mathbb{Z} $ dados por
\[
	U_1= \left\{ k\in\mathbb{Z}\colon k>0 \right\}
	\quad\text{y}\quad 
	U_2=\left\{ k\in\mathbb{Z}\colon k<0 \right\}.
\]

\reversemarginpar\marginnote{¿Es este contraejemplo válido?}
Con estos subconjuntos se tiene que
$\mathbb{Z}\setminus U_1$ son todos los enteros
negativos con el cero y, análogamente, $
\mathbb{Z}\setminus U_2 $ son todos los enteros
positivos con el cero, de donde se sigue que $ U_1,U_2
$ pertenecen a $ \mathcal{T}_\infty $. Pero entonces
\[
	\mathbb{Z}\setminus \big(U_1\cup U_2\big)= 
	\mathbb{Z}\setminus U_1\cap \mathbb{Z}\setminus U_2 
		= \left\{ 0 \right\},
\]
y $ U_1\cup U_2 $ \emph{no pertenece} a $ \mathcal{T}_\infty $.

\section*{Página 109, ejercicio 2}

No es cierto, en general, que $ f(x) $ sea un punto
límite de $ f(A) $. Consideremos el siguiente
contraejemplo.

%\reversemarginpar\marginnote{¿Este contraejemplo es
%válido?}
\normalmarginpar\marginnote{Véase el teorema 18.2, pag
105}
Sea $ f\colon X\to Y $ definida por $ f(x)=k, $ para
todo $x\in X,$ con $ k\in Y $ una constante fija.
Entonces esta función es continua, pero si $ A $ es un
subconjunto de $ X $ entonces $ f(A)=k $ y para todo $
a\in A $ se tiene que $ f(a)=k $ no es un punto límite
de $ A $ puesto que su todo entorno de $ f(a) $
intersecta a $ f(A) $ justamente en el punto $ k=f(a). $

\section*{Página 109, ejercicio 3}

Este ejercicio consta de dos partes.

\paragraph{Parte a}%
Supongamos que la función identidad $ \id $ es continua.
Entonces para todo elemento $ A $ de $ \mathcal{T} $
se tiene que $ \id^{-1}(A) $ pertenece a $ \mathcal{T}'.
$ Pero la inversa de la función identidad es ella
misma por lo que $ \id^{-1}(A)=A $ y $
\mathcal{T}\subset\mathcal{T}', $ como se buscaba.

Supongamos ahora que $ \mathcal{T}' $ es más fina que $
\mathcal{T}$. Entonces todo elemento de $ \mathcal{T} $
esta en $ \mathcal{T}' $. Como $ \id^{-1}(A)=A $, para
todo $A\in \mathcal{T}, $ y $ A\in\mathcal{T}' $ se
sigue que la imagen inversa de abiertos en $
\mathcal{T} $ son abiertos en $ \mathcal{T}' $ y la
función $ \id $ es continua.

\paragraph{Parte b}%
\label{par:parte_b}

Supongamos que $\id$ es un homeomorfismo, entonces la
función $\id$ es continua y por la parte anterior se
tiene la contención $\mathcal{T}\subset\mathcal{T}'$,
queda mostrar que $\mathcal{T}'\subset\mathcal{T}.$
Esto último en efecto es así puesto que las imágenes directas
de conjuntos abiertos en $\mathcal{T}'$ son abiertos en
$\mathcal{T}$, por ser $\id$ homeomorfismo, y la
sobreyectividad de $\id$ implica que \emph{todo} abierto
de $\mathcal{T}'$ es abierto en $\mathcal{T}$, es
decir, que $\mathcal{T}'\subset\mathcal{T}$.

Supongamos ahora que $\mathcal{T}'=\mathcal{T}$. Por la
parte anterior se tiene que $\id$ es continua, pero como la
inversa de la identidad es ella misma se sigue
inmediatamente que $\id^{-1}$ también es continua y que
$\id$ es un homeomorfismo.

\section*{Página 109, ejercicio 4}

La función $f$ es inyectiva puesto que
\[
	f(x_1)=f(x_2)
\]
implica que
\[
	(x_1,y_0)=(x_2,y_0)
\]
y esto, por definición del producto cartesiano, implica
que $x_1=x_2$. Un razonamiento análogo nos dice que la
función $g$ también es inyectiva.

Entonces, si restringimos el rango de $f$ y $g$ a sus
imágenes directas obtenemos dos funciones $f',g'$
biyectivas. Para ver que $f,g$ son inmersiones solo
hace falta ver que $f',g'$ y sus inversas son
continuas.

Para ver que $f'$ es continua, notemos que esta función
se puede escribir de la forma
\[
	f'=(\id_X,\gamma),
\]
donde $\id_X$ es la identidad de $X$ y $\gamma:X\to Y$ 
es la función constantemente igual a $y_0$.
\normalmarginpar\marginnote{Véase el teorema 18.4, pag
108}
Entonces la continuidad de $f'$ se sigue de la
continuidad de estas dos funciones. Un argumento
análogo sirve para la función $g'$ y por lo tanto
$f',g'$ son ambas continuas.

Al igual que antes, una vez que veamos que $f'^{-1}$ es
continua un argumento completamente análogo nos dará
como resultado que $g'^{-1}$ es continua. Para ver la
continuidad de $f'^{-1}$ notemos que esta función esta
dada, al fijar un $y_0\in Y$, por
\[
	f'^{-1}=\pi	
\]
donde $\pi$ es la proyección canónica de $X\times\left\{ y_0 \right\}$
sobre $X$. Entonces la continuidad de $f'^{-1}$ se
sigue de la continuidad de $\pi$.

Tenemos entonces que $f',g'$ y sus inversas son
continuas, por lo que $f,g$ son inmersiones.
\end{document}
