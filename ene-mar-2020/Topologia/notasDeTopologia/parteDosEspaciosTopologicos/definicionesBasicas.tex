\chapter{Definiciones Básicas}%
\label{cha:definiciones_básicas}

La definición de una topología que se dará a
continuación es una de muchas posibles, diferentes
autores\footnote{Fréchet, Hausdorff, entre otros.} han
propuesto diferentes manera de definirlas. La forma en
que se definen en este texto es la manera
\emph{estándar} de la cual se pueden obtener todos los
casos de topologías interesantes en los conjuntos
interesantes.

Aunque la definición de una topología pueda parecer
abstracta, su significado se irá haciendo más claro
cuanto más se use.

\begin{defi} Una \Keyword{topología} en un conjunto $X$ es
	una colección $\mathcal{T}$ de subconjuntos de
	$X$ (un subconjunto del conjunto de partes de
	$X$) que posee las siguientes propiedades:
	\begin{enumerate} \item El conjunto vacío y $X$
		pertenecen a $ \mathcal{T}$
	\item La
		union arbitraria de subconjuntos de $
		\mathcal{T} $ está en $ \mathcal{T} $.
	\item Las intersecciones finitas de elementos
		de $ \mathcal{T} $ estan en $
		\mathcal{T} $.  \end{enumerate} 
	Un
	conjunto $ X $ junto con una topología $
\mathcal{T} $ definida en $ X $ es llamado un
\emph{espacio topológico}.  \end{defi}

Si $ X $ es un espacio topológico con una topología $
\mathcal{T} $, diremos que un subconjunto $ U $ de $ X
$ es \emph{abierto} si $ U\in \mathcal{T} $. Dicho esto
la definición anterior puede enunciarse nuevamente en
términos de conjuntos abiertos: 
\begin{itemize} \item
	Tanto el vacío como $ X $ son abiertos.  
	\item
	Las uniones arbitrarias e intersecciones
	finitas de conjuntos abiertos dan conjuntos
	abiertos.  \end{itemize}

A partir de ahora se usará la frase `es un conjunto
abierto' como sinónimo de `pertenece a la topología $
\mathcal{T} $' a no ser que se diga lo contario.

\begin{ejem} 
	Si $ X $ es un conjunto, entonces el
	conjunto $ \mathcal{P}(X) $ de las
	\emph{partes} de $ X $ es una topología. Esta
	topología, en la que todos los conjuntos son
	abiertos, se conoce como la topología
	\emph{discreta}.  \end{ejem}

\begin{ejem} Si
	consideramos como abiertos solamente a los
	subconjuntos $ \emptyset,X $ de un conjunto $ X
	$ cualquiera entonces obtenemos una topología
	sobre $ X $ llamada \emph{indiscreta} o
	trivial.  \end{ejem}

\begin{ejem} Sean $ X $ un
	conjunto y $ \mathcal{T}_f $ la colección de
	todos los subconjuntos $ U $ de $ X $ tales que
	$ X\setminus U $ es finito o es todo $ X $.
	Entonces $ \mathcal{T}_f $ es una topología,
	llamada la \emph{topología complemento finito}.

	Tanto $ X $ como el vacío pertenecen a $
	\mathcal{T}_f $ debido a que $ X\setminus X $
	es finito y $ X\setminus\emptyset $ es todo $ X
	$. Si $\{ U_\alpha \}$ es una familia indexada
	de elementos de $ \mathcal{T}_f $ no nulos,
	entonces\footnotemark \[ X\setminus\bigcup
	U_\alpha = \bigcap (X\setminus U_\alpha).  \]
	\footnotetext{Las igualdades de este ejemplo se
		siguen todas de las leyes de \emph{De
	Morgan}.} donde el lado derecho de la igualdad
	es conjunto finito puesto que los $ X\setminus
	U_\alpha $ son finitos. Se sigue que las
	uniones arbitrarias de elementos de $
	\mathcal{T}_f $ pertenecen a $ \mathcal{T}_f $.

	Sean $ U_1,\dots,U_n $ una cantidad finita de
	elementos no vacíos de $ \mathcal{T}_f $.
	Entonces \[ X\setminus\bigcap_{i=1}^n U_i =
		\bigcup_{i=1}^n (X\setminus U_i).  \]
	Donde el lado derecho de la igualdad es finito
por ser la unión de conjuntos finitos.  \end{ejem}




